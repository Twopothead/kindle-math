\documentclass[kindlepaper]{BHCexam4kindle}
\usepackage{lipsum}
\begin{document}
\printanswers % 我要打印答案
	\biaoti{2018真题}
	\fubiaoti{}
	\maketitle
	\begin{questions}
		\qs 下列函数中, 在 x D 0 处不可导的是\xx
		\fourch{$f(x)=|x| \sin |x|$}{$f(x)=|x| \sin \sqrt{|x|}$}
		{$f(x)=\cos |x|$}{$f(x)=\cos \sqrt{|x|}$}
		\begin{solution}
			A,B,C 可直接验证可导, D 根据导数的定义可得$f_{+}^{\prime}(0)=-\frac{1}{2}, f_{-}^{\prime}(0)=\frac{1}{2}$.
		\end{solution}

		\qs 过点 $(1,0,0)$ 与$(0,1,0)$ 且与曲面 $z=x^{2}+y^{2}$相切的平面方程为\xx
		\fourch{z=0与x+y-z=1}
			   {z=0与2x+2y-z=0}
			   {y=x与x+y-z=1}
			   {y=x与2x+2y-z=2}
		\begin{solution}
			过点$(1,0,0)$ 与$(0,1,0)$ 且与已知曲面相切的平面只有两个, 显然 $z = 0$ 与曲面$z=x^{2}+y^{2}$
			相切, 故排除 C,D. 曲面 $z=x^{2}+y^{2}$ 的法向量为 $(2 x, 2 y,-1)$, 对于 A 选项,$x+y-z=1$
			的法向量为 $(1,1,-1)$, 可得$x=\frac{1}{2}, y=\frac{1}{2}$ 代入 $z=x^{2}+y^{2}$ 和 $x+y-z=1$中 z 不相等,
			排除 A, 故选 B.
		\end{solution}

		\qs $\sum_{n=0}^{\infty}(-1)^{n} \frac{2 n+3}{(2 n+1) !}=$\xx.
		\twoch{$\sin 1+\cos 1$}{$2 \sin 1+\cos 1$}{$2 \sin 1+2 \cos 1$}{$3 \sin 1+2 \cos 1$}
		\begin{solution}
			
			利用 $sin x$ 与 $cos x$ 的麦克劳林级数可得\\
		$\begin{aligned} \sum_{n=0}^{\infty}(-1)^{n} \frac{2 n+3}{(2 n+1) !} 
		&=\sum_{n=0}^{\infty}(-1)^{n} \frac{(2 n+1)+2}{(2 n+1) !} \end{aligned}$\\
		$
=\sum_{n=0}^{\infty}(-1)^{n} \frac{1}{(2 n) !}+\sum_{n=0}^{\infty}(-1)^{n} \frac{2}{(2 n+1) !} 
\\=2 \sin 1+\cos 1 $\\
			因此选 B.
		\end{solution}

		\qs 设$M=\int_{-\frac{\pi}{2}}^{\frac{\pi}{2}} \frac{(1+x)^{2}}{1+x^{2}} \mathrm{d} x, N=\int_{-\frac{\pi}{2}}^{\frac{\pi}{2}} \frac{1+x}{\mathrm{e}^{x}} \mathrm{d} x, K=\int_{-\frac{\pi}{2}}^{\frac{\pi}{2}}(1+\sqrt{\cos x}) \mathrm{d} x$,
		则\xx
		\twoch{$M>N>K$}{$M>K>N$}{$K>M>N$}{$N>M>K$}
		\begin{solution}
			利用对称性可以计算$M=\int_{-\frac{\pi}{2}}^{\frac{\pi}{2}} \frac{(1+x)^{2}}{1+x^{2}} \mathrm{d} x=\int_{-\frac{\pi}{2}}^{\frac{\pi}{2}}\left(1+\frac{2 x}{1+x^{2}}\right) \mathrm{d} x=\pi$,
			另外比较被积函数与 1 的大小关系易见$K>\pi=M>N$
		\end{solution}

		\qs 下列矩阵中, 与矩阵$\left(\begin{array}{lll}{1} & {1} & {0} \\ {0} & {1} & {1} \\ {0} & {0} & {1}\end{array}\right)$相似的为\xx
			\twoch{$\left(\begin{array}{ccc}{1} & {1} & {-1} \\ {0} & {1} & {1} \\ {0} & {0} & {1}\end{array}\right)$}
				{$\left(\begin{array}{ccc}{1} & {0} & {-1} \\ {0} & {1} & {1} \\ {0} & {0} & {1}\end{array}\right)$}
					{$\left(\begin{array}{ccc}{1} & {1} & {-1} \\ {0} & {1} & {0} \\ {0} & {0} & {1}\end{array}\right)$}
						{$\left(\begin{array}{ccc}{1} & {0} & {-1} \\ {0} & {1} & {0} \\ {0} & {0} & {1}\end{array}\right)$}
		\begin{solution}
			易知题中矩阵均有 3 重特征值 1. 若矩阵相似, 则不同特征值对应矩阵$\lambda \boldsymbol{E}-\boldsymbol{A}$
			的秩相等, 即$E-A$的秩相等, 选 A.
		\end{solution}

		\qs 设 A,B 为 n 阶矩阵, 记 $r(X)$ 为矩阵 X 的秩, $(\boldsymbol{X} \boldsymbol{Y})$ 表示分块矩阵, 则\xx
		\fourch{$r(\boldsymbol{A} \boldsymbol{A} \boldsymbol{B})=r(\boldsymbol{A})$}
		{$r(\boldsymbol{A} \boldsymbol{B} \boldsymbol{A})=r(\boldsymbol{A})$}
		{$r(\boldsymbol{A} \boldsymbol{B})=\max \{r(\boldsymbol{A}), r(\boldsymbol{B})\}$}
		{$r(\boldsymbol{A} \boldsymbol{B})=r\left(\boldsymbol{A}^{\mathrm{T}} \boldsymbol{B}^{\mathrm{T}}\right)$}
		\begin{solution}
			对于 A, 有$r(\boldsymbol{A} \boldsymbol{A} \boldsymbol{B})=r(\boldsymbol{A}(\boldsymbol{E} \boldsymbol{B}))$,
			且 $(\boldsymbol{E} \boldsymbol{B})$为行满秩的矩阵, 则 $r(\boldsymbol{A} \boldsymbol{A} \boldsymbol{B})=r(\boldsymbol{A})$,
			即选A.B错误,反例举$A=\left(\begin{array}{ll}{1} & {0} \\ {0} & {0}\end{array}\right), B=\left(\begin{array}{ll}{1} & {0} \\ {1} & {1}\end{array}\right)$. C 错误,
				$r(\boldsymbol{A} \boldsymbol{B}) \geqslant \max \{r(\boldsymbol{A}), r(\boldsymbol{B})\}$,反例举$A=\left(\begin{array}{ll}{1} & {0} \\ {0} & {0}\end{array}\right), \boldsymbol{B}=\left(\begin{array}{ll}{0} & {0} \\ {0} & {1}\end{array}\right)$. D 错误, 反例取$A=\left(\begin{array}{ll}{1} & {0} \\ {0} & {0}\end{array}\right), \boldsymbol{B}=\left(\begin{array}{ll}{0} & {0} \\ {1} & {0}\end{array}\right)$.
		\end{solution}

		\qs 设随机变量 X的概率密度$f(x)$满足$f(1+x)=f(1-x)$,且$\int_{0}^{2} f(x) \mathrm{d} x=0.6$,则
		$P(X<0)=$\xx
		\onech{0.2}{0.3}{0.4}{0.6}
		\begin{solution}
			由$f(1+x)=f(1-x)$知$f(x)$关于 $x = 1$ 对称, 则\\
			$\int_{0}^{1} f(x) \mathrm{d} x=\int_{1}^{2} f(x) \mathrm{d} x=\frac{1}{2} \int_{0}^{2} f(x) \mathrm{d} x=0.3$.\\
			于是$P\{X<0\}=\int_{-\infty}^{0} f(x) \mathrm{d} x=\int_{-\infty}^{1} f(x) \mathrm{d} x-\int_{0}^{1} f(x) \mathrm{d} x=0.5-0.3=0.2$. 选 A.
		\end{solution}

		\qs 给定总体$X \sim N\left(\mu, \sigma^{2}\right), \sigma^{2}$ 已知, 
		给定样本$X_{1}, X_{2}, \cdots, X_{n}$, 对总体均值 $\mu$ 进行检验,
		令 $H_{0} : \mu=\mu_{0}, H_{1} : \mu \neq \mu_{0}$ , 则\xx\\
		\fourch
		{若显著性水平 $\alpha=0.05$ 时拒绝$H_{0}$ , 则  $\alpha=0.01$ 时必拒绝$H_{0}$}
		{若显著性水平 $\alpha=0.05$ 时接受$H_{0}$ , 则  $\alpha=0.01$ 时必拒绝$H_{0}$}
		{若显著性水平 $\alpha=0.05$ 时拒绝$H_{0}$ , 则  $\alpha=0.01$ 时必接受$H_{0}$}
		{若显著性水平 $\alpha=0.05$ 时接受$H_{0}$ , 则  $\alpha=0.01$ 时必接受$H_{0}$}
		\begin{solution}
			显著性水平为$\alpha$的假设检验的接受域就是置信水平为 $1-\alpha$的置信区间, 
			当 $\alpha$ 变小时,置信区间会变大, 也就是接受域变大, 因此选 D.
		\end{solution}

		\qs $\lim _{x \rightarrow 0}\left(\frac{1-\tan x}{1+\tan x}\right)^{\frac{1}{\sin (k x)}}=\mathrm{e}$,则k=\tk
		\begin{solution}
			原极限为 $1^{\infty}$ 型, 故恒等变形为\\
			$\lim _{x \rightarrow 0}\left(1+\frac{-2 \tan x}{1+\tan x}\right)^{\frac{1+\tan x}{2 \tan x} \frac{-2 \tan x}{(1+\tan x) \sin (k x)}}\\
			=\exp \left(\lim _{x \rightarrow 0} \frac{-2 \tan x}{(1+\tan x) \sin (k x)}\right)=\mathrm{e}^{-\frac{2}{k}}$.\\
			所以$-\frac{2}{k}=1, k=-2$.

		\end{solution}

		\qs 设函数 $f(x)$ 具有 2 阶连续导数,
		若曲线 $y=f(x)$的过点$(0,0)$, 且与曲线$y=2^{x}$在点$(1,2)$ 处相切, 则$\int_{0}^{1} x f^{\prime \prime}(x) \mathrm{d} x=$\xx\\
		\begin{solution}
			由题意知 $f(0)=0, f(1)=2, f^{\prime}(1)=2^{x} \ln \left.2\right|_{x=1}=2 \ln 2$. 
			由分部积分公式, 原积分等于$x f^{\prime}\left.(x)\right|_{0} ^{1}-\int_{0}^{1} f^{\prime}(x) \mathrm{d} x$\\
			$=2 \ln 2-2$.
		\end{solution}

		\qs 设$\boldsymbol{F}(x, y, z)=x y \boldsymbol{i}-y z \boldsymbol{j}+x z \boldsymbol{k}$,求$\operatorname{rot} \boldsymbol{F}(1,1,0)=$\xx\\
		\begin{solution}
		由旋度定义$\operatorname{rot} \boldsymbol{F}=\left|\begin{array}{ccc}{\boldsymbol{i}} & {\boldsymbol{j}} & {\boldsymbol{k}} \\ {\frac{\partial}{\partial x}} & {\frac{\partial}{\partial y}} & {\frac{\partial}{\partial z}} \\ {x y} & {-y z} & {x z}\end{array}\right|=y \boldsymbol{i}-z \boldsymbol{j}-x \boldsymbol{k}$\\
			, 可知$\operatorname{rot} \boldsymbol{F}(1,1,0)=\boldsymbol{i}-\boldsymbol{k}$.
		\end{solution}


		\qs	设 L 为球面$x^{2}+y^{2}+z^{2}=1$ 与平面$x+y+z=0$的交线, 则$\oint_{L} x y d s=$\tk
		\begin{solution}
			由对称性得\\
		$\begin{aligned} \oint_{L} x y \mathrm{d} s &=\frac{1}{3} \oint_{L}(x y+y z+x z) \mathrm{d} s \\ &=\frac{1}{6} \oint_{L}\left[(x+y+z)^{2}-\left(x^{2}+y^{2}+z^{2}\right)\right] \mathrm{d} s \\ &=\frac{1}{6} \oint_{L}(-1) \mathrm{d} s=-\frac{\pi}{3} \end{aligned}$
		\end{solution}

		\qs 设 2 阶矩阵 A 有两个不同的特征值,  ̨$\alpha_{1}, \alpha_{2}$是 A 的线性无关的特征向量,
		$\mathbf{A}^{2}\left(\alpha_{1}+\right.\boldsymbol{\alpha}_{2} )=\boldsymbol{\alpha}_{1}+\boldsymbol{\alpha}_{2}$
		, 则$|A|=$\tk
		\begin{solution}
			由 $\alpha_{1}, \alpha_{2}$是 A 的线性无关的特征向量, 则$\alpha_{1}, \alpha_{2}$ 是$A^{2}$ 的线性无关的特征向量. 又
			$A^{2}\left(\alpha_{1}+\alpha_{2}\right)=\alpha_{1}+\alpha_{2}, \alpha_{1}+\alpha_{2}$
			也是 $A^{2}$ 的特征向量, 则$A^{2}$  有二重特征值 1. 又 A 有两个
			不同的特征值, 则其特征值为$-1,1$, 故 $|A|=-1$.
		\end{solution}

		\qs .设随机事件 A 与 B 相互独立, A 与 C 相互独立, $B C=\varnothing$. 若\\
		$P(A)=P(B)=\frac{1}{2} P(A C | A B \cup C)=\frac{1}{4}$,则$P(C)=$\tk
		\begin{solution}
			因为$B C=\varnothing, P(B C)=0, $故$ P(A B C)=0$\\
		$\begin{aligned} P(A C | A B \cup C) &=\frac{P[(A B C) \cup(A C)]}{P(A B \cup C)} \\ &=\frac{P(A C)}{P(A B)+P(C)-P(A B C)} \\ &=\frac{P(A) P(C)}{P(A) P(B)+P(C)}=\frac{1}{4} \end{aligned}$
			解得$P(C)=\frac{1}{4}$
		\end{solution}

		\qs 求不定积分$\int \mathrm{e}^{2 x} \arctan \sqrt{\mathrm{e}^{x}-1} \mathrm{d} x$
		\begin{solution}
			利用分部积分法\\
		$ \int \mathrm{e}^{2 x} \arctan \sqrt{\mathrm{e}^{x}-1} \mathrm{d} x $\\
	$=\frac{1}{2} \int \arctan \sqrt{\mathrm{e}^{x}-1} \mathrm{d}\left(\mathrm{e}^{2 x}\right) 
	\\ =\frac{1}{2} \mathrm{e}^{2 x} \arctan \sqrt{\mathrm{e}^{x}-1}-\frac{1}{4} \int \frac{\mathrm{e}^{2 x}}{1+\mathrm{e}^{x}-1} \frac{\mathrm{e}^{x}}{2 \sqrt{\mathrm{e}^{x}-1}} \mathrm{d} x 
\\ =\frac{1}{2} \mathrm{e}^{2 x} \arctan \sqrt{\mathrm{e}^{x}-1}-\frac{1}{4} \int \frac{\mathrm{e}^{x}}{\sqrt{\mathrm{e}^{x}-1}} \mathrm{d}\left(\mathrm{e}^{x}\right) \\ &=\frac{1}{2} \mathrm{e}^{2 x} \arctan \sqrt{\mathrm{e}^{x}-1}-\frac{1}{4} \int \frac{\mathrm{e}^{x}}{\sqrt{\mathrm{e}^{x}-1}} \mathrm{d}\left(\mathrm{e}^{x}\right)$\\
			其中\\
		$\begin{aligned} \int \frac{\mathrm{e}^{x}}{\sqrt{\mathrm{e}^{x}-1}} \mathrm{d}\left(\mathrm{e}^{x}\right) &=\int \frac{t}{\sqrt{t-1}} \mathrm{d} t=\int \frac{t-1+1}{\sqrt{t-1}} \mathrm{d} t \\ &=\int \sqrt{t-1} \mathrm{d} t+\int \frac{\mathrm{d} t}{\sqrt{t-1}} \\ &=\frac{2}{3}(t-1)^{\frac{3}{2}}+2 \sqrt{t-1}+C \\ &=\frac{2}{3}\left(\mathrm{e}^{x}-1\right)^{\frac{3}{2}}+2 \sqrt{\mathrm{e}^{x}-1}+C \end{aligned}$\\
			故$\int \mathrm{e}^{2 x} \arctan \sqrt{\mathrm{e}^{x}-1} \mathrm{d} x=\frac{1}{2} \mathrm{e}^{2 x} \arctan \sqrt{\mathrm{e}^{x}-1}-\frac{1}{6}\left(\mathrm{e}^{x}-1\right)^{\frac{3}{2}}-\frac{1}{2} \sqrt{\mathrm{e}^{x}-1}+C_{1}$.
		\end{solution}

		\qs 将长为 2 m 的铁丝分成三段, 依次围成圆、正方形与正三角形, 三个图形的面积之
		和是否存在最小值?若存在, 求出最小值.
		\begin{solution}
			设分成的三段依次为 x, y,z, 则$x+y+z=2$, 依次围成的圆的半径、正方形的边长与
		正三角形的边长分别为$\frac{x}{2 \pi}, \frac{y}{4}, \frac{z}{3}$, 因此三个面积的和为\\
		$S=\pi\left(\frac{x}{2 \pi}\right)^{2}+\left(\frac{y}{4}\right)^{2}+\frac{\sqrt{3}}{4}\left(\frac{z}{3}\right)^{2}=\frac{x^{2}}{4 \pi}+\frac{1}{16} y^{2}+\frac{\sqrt{3}}{36} z^{2}$
		\end{solution}
		\begin{solution}
			令$f(x, y, z, \lambda)=\frac{x^{2}}{4 \pi}+\frac{1}{16} y^{2}+\frac{\sqrt{3}}{36} z^{2}+\lambda(x+y+z-2)$
			, 首先求驻点. 由方程$\left\{\begin{array}{l}{f_{x}^{\prime}=\frac{x}{2 \pi}+\lambda=0} \\ {f_{y}^{\prime}=\frac{y}{8}+\lambda=0} \\ {f_{z}^{\prime}=\frac{\sqrt{3}}{18} z+\lambda=0} \\ {x+y+z=2}\end{array}\right.$
				可得$\left\{\begin{array}{l}{x=\frac{2 \pi}{\pi+4+3 \sqrt{3}}} \\ {y=\frac{8}{\pi+4+3 \sqrt{3}}} \\ {z=\frac{6 \sqrt{3}}{\pi+4+3 \sqrt{3}}}\end{array}\right.$
					, 并且$H f=\operatorname{diag}\left\{\frac{1}{2 \pi}, \frac{1}{8}, \frac{\sqrt{3}}{18}\right\}$正定, 这就是面积和的最小值点, 此时最小面积为$S_{\min }=\frac{1}{\pi+4+3 \sqrt{3}} \mathrm{m}^{2}$
		\end{solution}
		\begin{solution}
			由柯西不等式\\
			$\left(\frac{x^{2}}{4 \pi}+\frac{1}{16} y^{2}+\frac{\sqrt{3}}{36} z^{2}\right)\left(4 \pi+16+\frac{36}{\sqrt{3}}\right) \geqslant(x+y+z)^{2}=4$\\
			, 因此当$\frac{x}{2 \pi}=\frac{y}{16}=\frac{z}{12 \sqrt{3}}$即$\left\{\begin{aligned} x &=\frac{2 \pi}{\pi+4+3 \sqrt{3}} \\ y &=\frac{8}{\pi+4+3 \sqrt{3}} \\ z &=\frac{6 \sqrt{3}}{\pi+4+3 \sqrt{3}} \end{aligned}\right.$
				时,$S_{\min }=\frac{1}{\pi+4+3 \sqrt{3}} \mathrm{m}^{2}$.
		\end{solution}

		\qs 设$\Sigma$ 是曲面$x=\sqrt{1-3 y^{2}-3 z^{2}}$ 的前侧, 计算曲面积分\\
		$\iint_{\Sigma} x \mathrm{d} y \mathrm{d} z+\left(y^{3}+z\right) \mathrm{d} z \mathrm{d} x+z^{3} \mathrm{d} x \mathrm{d} y$.
		\begin{solution}
			取曲面 $\Sigma_{1} : x=0,3 y^{2}+3 z^{2} \leqslant 1$, 
			法向量方向指向 x 轴负向. 记$\Omega$ 为$\Sigma$和 $\Sigma_{1}$ 所围成的区域, 则\\
		$\begin{aligned} & \iint_{\Sigma} x \mathrm{d} y \mathrm{d} z+\left(y^{3}+z\right) \mathrm{d} z \mathrm{d} x+z^{3} \mathrm{d} x \mathrm{d} y \\=& \iint_{\Sigma+\Sigma_{1}} x \mathrm{d} y \mathrm{d} z+\left(y^{3}+z\right) \mathrm{d} z \mathrm{d} x+z^{3} \mathrm{d} x \mathrm{d} y-\iint_{\Sigma_{1}} x \mathrm{d} y \mathrm{d} z+\left(y^{3}+z\right) \mathrm{d} z \mathrm{d} x+z^{3} \mathrm{d} x \mathrm{d} y \end{aligned}$.\\
			由高斯公式得\\
		$ \iint_{\Sigma+\Sigma_{1}} x \mathrm{d} y \mathrm{d} z+\left(y^{3}+z\right) \mathrm{d} z \mathrm{d} x+z^{3} \mathrm{d} x \mathrm{d} y 
		\\=& \iiint_{\Omega}\left(1+3 y^{2}+3 z^{2}\right) \mathrm{d} V
		\\=\iiint_{\Omega} \mathrm{d} V+3 \iint_{3 y^{2}+3 z^{2}<1}\left(y^{2}+z^{2}\right) \sqrt{1-3 y^{2}-3 z^{2}} \\\mathrm{dydz} $\\
			$=\frac{1}{2} \cdot \frac{4 \pi}{3} \cdot \frac{\sqrt{3}}{3} \cdot \frac{\sqrt{3}}{3}+3 \int_{0}^{2 \pi} \mathrm{d} \theta \int_{0}^{\frac{1}{\sqrt{3}}} r^{2} \sqrt{1-3 r^{2}} r \mathrm{d} r=\frac{14 \pi}{45}$\\
			而$\iint_{\Sigma_{1}} x \mathrm{d} y \mathrm{d} z+\left(y^{3}+z\right) \mathrm{d} z \mathrm{d} x+z^{3} \mathrm{d} x \mathrm{d} y=0$, 所以
			$\iint_{\Sigma} x \mathrm{d} y \mathrm{d} z+\left(y^{3}+z\right) \mathrm{d} z \mathrm{d} x+z^{3} \mathrm{d} x \mathrm{d} y$
			$=\frac{14 \pi}{45}$.\\
		\end{solution}

		\qs 已知微分方程$y^{\prime}+y=f(x)$, 其中$f(x)$ 是 $\mathbb{R}$ 上的连续函数.
		\begin{parts}
			\part 当 $f(x)=x$ 时, 求微分方程的通解.
			\part 若 $f(x)$ 是周期为 T 的函数, 证明: 方程存在唯一的以 T 为周期解.	
		\end{parts}

		\begin{solution}
			等式两边乘以$e^{x}$可得$\left(\mathrm{e}^{x} y\right)^{\prime}=\mathrm{e}^{x} f(x)$, 
			通解可表示为$y(x)=\mathrm{e}^{-x}\left(\int_{0}^{x} f(t) \mathrm{e}^{t} \mathrm{d} t+C\right)$.\\
			现在 $f(x+T)=f(x)$, 则\\$y(x+T)$\\
		$\begin{aligned} &=\mathrm{e}^{-x-T}\left(\int_{0}^{x+T} f(t) \mathrm{e}^{t} \mathrm{d} t+C\right) \\ &=\mathrm{e}^{-x-T}\left(\int_{0}^{T} f(t) \mathrm{e}^{t} \mathrm{d} t+\int_{T}^{T+x} f(t) \mathrm{e}^{t} \mathrm{d} t+C\right) \\ &=\mathrm{e}^{-x-T}\left(\int_{0}^{T} f(t) \mathrm{e}^{t} \mathrm{d} t+\int_{T}^{x} f(u+T) \mathrm{e}^{u+T} \mathrm{d} u+C\right) \\ &=\mathrm{e}^{-x-T}\left(\left(\int_{0}^{T} f(t) \mathrm{e}^{t} \mathrm{d} t+C_{0}^{x} f(u) \mathrm{e}^{u+T} \mathrm{d} u+C\right)\right.\\ &=\mathrm{e}^{-x}\left(\left(\int_{0}^{T} f(t) \mathrm{e}^{t} \mathrm{d} t+C\right) \mathrm{e}^{-T}+\int_{0}^{x} f(u) \mathrm{e}^{u} \mathrm{d} u\right) \end{aligned}$
					\\要使得这个解是周期函数, 则$y(x+T)=y(x)$,即满足$\left(\int_{0}^{T} f(t) \mathrm{e}^{t} \mathrm{d} t+C\right) \mathrm{e}^{-T}=C$,
			由此解得$C=\frac{\int_{0}^{T} f(t) \mathrm{e}^{t} \mathrm{d} t}{\mathrm{e}^{T}-1}$
			, 因此$y=\mathrm{e}^{-x}\left(\int_{0}^{x} f(t) \mathrm{e}^{t} \mathrm{d} t+\frac{\int_{0}^{T} f(t) \mathrm{e}^{t} \mathrm{d} t}{\mathrm{e}^{T}-1}\right)$
			就是唯一的周期函数解.
		\end{solution}

		\qs 设数列$\left\{x_{n}\right\}$ 满足$x_{1}>0, x_{n} \mathrm{e}^{x_{n+1}}=\mathrm{e}^{x_{n}}-1(n=1,2, \cdots)$,
		证明$\left\{x_{n}\right\}$收敛并求$\lim _{n \rightarrow \infty} x_{n}$.
		\begin{solution}
			首先由$x_{1}>0, x_{n} \mathrm{e}^{x_{n+1}}=\mathrm{e}^{x_{n}}-1(n=1,2, \cdots)$
			归纳可知所有$x_{n}>0$. 考虑函数$f(x)=\mathrm{e}^{x}$, 由拉格朗日中值定理有\\
			$\mathrm{e}^{x_{n+1}}=\frac{\mathrm{e}^{x_{n}}-1}{x_{n}}=\frac{f\left(x_{n}\right)-f(0)}{x_{n}-0}=\mathrm{e}^{\xi_{n}}<\mathrm{e}^{x_{n}}, \xi_{n} \in\left(0, x_{n}\right)$.
			这就说明$x_{n}>x_{n+1}>0$, 因此$\left\{x_{n}\right\}$ 单调递减有下界, 故收敛. 设 $\lim _{n \rightarrow \infty} x_{n}=x \geqslant 0$
			, 在等式$x_{n} \mathrm{e}^{x_{n+1}}=\mathrm{e}^{x_{n}}-1$两边取极限得$x \mathrm{e}^{x}=\mathrm{e}^{x}-1$. 如果
			$x>0$,则$e^{x}=\frac{e^{x}-1}{x}<e^{x}$, 矛盾, 因此$\lim _{n \rightarrow \infty} x_{n}=x=0$.
		\end{solution}

		\qs 设实二次型$f\left(x_{1}, x_{2}, x_{3}\right)=\left(x_{1}-x_{2}+x_{3}\right)^{2}+\left(x_{2}+x_{3}\right)^{2}+\left(x_{1}+a x_{3}\right)^{2}$, 其中 a 是参数.
		\begin{parts}
			\part 求$f\left(x_{1}, x_{2}, x_{3}\right)=0$ 的解;
			\part 求 $f\left(x_{1}, x_{2}, x_{3}\right)$ 的规范形.
		\end{parts}
		\begin{solution}
			\begin{parts}

			\part 由 $f\left(x_{1}, x_{2}, x_{3}\right)=0$可得方程组\\
		$\left\{\begin{array}{l}{x_{1}-x_{2}+x_{3}=0} \\ {x_{2}+x_{3}=0} \\ {x_{1}+a x_{3}=0}\end{array}\right.$.
				对其系数矩阵进行初等行变换得\\
			$\left(\begin{array}{ccc}{1} & {-1} & {1} \\ {0} & {1} & {1} \\ {1} & {0} & {a}\end{array}\right) \rightarrow\left(\begin{array}{ccc}{1} & {-1} & {1} \\ {0} & {1} & {1} \\ {0} & {0} & {a-2}\end{array}\right)$
				如果 $a=2$, 则方程组的通解为 $\left(x_{1}, x_{2}, x_{3}\right)^{\mathrm{T}}=c(-2,-1,1)^{\mathrm{T}}$
				 . 如果$a=2$, 则方程组只有零解$\left(x_{1}, x_{2}, x_{3}\right)^{\mathrm{T}}=(0,0,0)^{\mathrm{T}}$.
			\part
			如果 $a \neq 2$, 令\\$\left(\begin{array}{l}{y_{1}} \\ {y_{2}} \\ {y_{3}}\end{array}\right)=\left(\begin{array}{ccc}{1} & {-1} & {1} \\ {0} & {1} & {1} \\ {1} & {0} & {a}\end{array}\right)\left(\begin{array}{c}{x_{1}} \\ {x_{2}} \\ {x_{3}}\end{array}\right)=Q x$\\
				其中 Q 是可逆矩阵, 所以此时的规范形为$f\left(y_{1}, y_{2}, y_{3}\right)=y_{1}^{2}+y_{2}^{2}+y_{3}^{2}$.
				如果$a=2$, 配方得$f\left(x_{1}, x_{2}, x_{3}\right)$\\
				$=\left(x_{1}-x_{2}+x_{3}\right)^{2}+\left(x_{2}+x_{3}\right)^{2}+\left(x_{1}+2 x_{3}\right)^{2}$
$=2 x_{1}^{2}+2 x_{2}^{2}+6 x_{3}^{2}-2 x_{1} x_{2}+6 x_{1} x_{3}$\\
$=2\left(x_{1}-\frac{1}{2} x_{2}+\frac{3}{2} x_{3}\right)^{2}+\frac{3}{2}\left(x_{2}+x_{3}\right)^{2}$\\
				此时的规范形为$f\left(y_{1}, y_{2}, y_{3}\right)=y_{1}^{2}+y_{2}^{2}$.	
			\end{parts}
		\end{solution}

		\qs 已知 a 是常数, 且矩阵$A=\left(\begin{array}{ccc}{1} & {2} & {a} \\ {1} & {3} & {0} \\ {2} & {7} & {-a}\end{array}\right)$
		可经初等列变换化为矩阵$\boldsymbol{B}=\left(\begin{array}{ccc}{1} & {a} & {2} \\ {0} & {1} & {1} \\ {-1} & {1} & {1}\end{array}\right)$.
			\begin{parts}
				\part 求 a;
				\part 求满足$\boldsymbol{A P}=\boldsymbol{B}$ 的可逆矩阵$P$.
			\end{parts}
		\begin{solution}
			\begin{parts}
			\part 由于矩阵 A 可经过初等列变换化为矩阵 B, 因此 A 和 B 的列向量组等价. 则对增广矩阵做初等行变换得\\
		$(A B)=\left(\begin{array}{cccccc}{1} & {2} & {a} & {1} & {a} & {2} \\ {1} & {3} & {0} & {0} & {1} & {1} \\ {2} & {7} & {-a} & {-1} & {1} & {1}\end{array}\right) 
		\\ \rightarrow\left(\begin{array}{cccccc}{1} & {2} & {a} & {1} & {a} & {2} \\ {0} & {1} & {-a} & {-1} & {1-a} & {-1} \\ {0} & {0} & {0} & {0} & {a-2} & {0}\end{array}\right)$\\
			因此 a = 2.
			\part 问题等价于解矩阵方程 $A X=B$, 也就是解三个非齐次线性方程组. 由 (1) 可得\\
		$\left(\begin{array}{ll}{A} & {B}\end{array}\right) \rightarrow$
		$\left(\begin{array}{llllll}{1} & {0} & {6} & {3} & {4} & {4} \\ {0} & {1} & {-2} & {-1} & {-1} & {-1} \\ {0} & {0} & {0} & {0} & {0} & {0}\end{array}\right)$\\
			解得$P=\left(\begin{array}{ccc}{-6 k_{1}+3} & {-6 k_{2}+4} & {-6 k_{3}+4} \\ {2 k_{1}-1} & {2 k_{2}-1} & {2 k_{3}-1} \\ {k_{1}} & {k_{2}} & {k_{3}}\end{array}\right)$\\$, k_{1}, k_{2}, k_{3}$
				为任意常数. 注意到 P 是可逆矩阵, 因此$|\boldsymbol{P}| \neq 0$, 这要求$k_{2} \neq k_{3}$.		
			\end{parts}
			\end{solution}
		
		\qs 已知随机变量 X 与 Y 相互独立, X 的概率分布为$P\{X=1\}=P\{X=-1\}=\frac{1}{2}, Y$服从参数为  的泊松分布, 令
		$Z=X Y$.
		\begin{parts}
			\part 求 $\operatorname{Cov}(X, Z)$;
			\part 求 Z 的概率分布.
		\end{parts}
		\begin{solution}
			\begin{parts}
				\part 直接计算可知 $E(X)=0, E\left(X^{2}\right)=1$, 而$Y \sim P(\lambda), E(Y)=\lambda$, 因此$\operatorname{Cov}(X, Z) $\\
				$=\operatorname{Cov}(X, X Y)=E\left(X^{2} Y\right)-E(X) E(X Y)$
				$=E\left(X^{2}\right) E(Y)-(E X)^{2} E(Y)=\lambda$
				\part 首先有$P(Z=k)$\\
				$=P(X=1) P(Z=k | X=1)+P(X=-1) P(Z=k | X=-1)$\\
				$=P(X=1) P(Y=k)+P(X=-1) P(Y=-k)$\\
				$=\frac{1}{2} P(Y=k)+\frac{1}{2} P(Y=-k)$\\
				当$k=1,2,3, \dots$时,$P\{Z=k\}=\frac{1}{2} P\{Y=k\}=\frac{\lambda^{k} \mathrm{e}^{-\lambda}}{2 k !}$;
				当$k=0$时,$P(Z=0)=P(Y=0)=\mathrm{e}^{-\lambda}$;
				当$k=-1,-2,-3, \cdots$时,$P\{Z=k\}=\frac{1}{2} P\{Y=-k\}=\frac{\lambda^{-k} \mathrm{e}^{-\lambda}}{2(-k) !}$.\\
				因此综上所述可得\\
			$P(Z=k)=\left\{\begin{array}{ll}{\frac{\lambda^{|k|} \mathrm{e}^{-\lambda}}{2|k| !},} & {k=\pm 1, \pm 2, \cdots} \\ {\mathrm{e}^{-\lambda},} & {k=0}\end{array}\right.$.
			\end{parts}
		\end{solution}

		\qs 设总体 X 的概率密度为\\
		$f(x, \sigma)=\frac{1}{2 \sigma} \mathrm{e}^{-\frac{|x|}{\sigma}},-\infty<x<+\infty$,
		其中$\sigma \in(0,+\infty)$ 为未知参数, $X_{1}, X_{2}, \cdots, X_{n}$ 为来自总体 X 的简单随机样本, 记 $\sigma$
		的最大似然估计量为.
		\begin{parts}
			\part 求$\hat{\sigma}$;
			\part 求$E(\hat{\sigma}), D(\hat{\sigma})$.
		\end{parts}
		\begin{solution}
			\begin{parts}
				\part 设 $X_{1}, X_{2}, \cdots, X_{n}$ 对应的样本值为$x_{1}, x_{2}, \cdots, x_{n}$ , 则似然函数为\\
				$L(\sigma)=\prod_{i=1}^{n} f\left(x_{i}, \sigma\right)=2^{-n} \sigma^{-n} \mathrm{e}^{-\frac{\sum_{i=1}^{n}\left|x_{i}\right|}{\sigma}}$,\\
				取对数得$\ln L(\sigma)=-n \ln 2-n \ln \sigma-\frac{1}{\sigma} \sum_{i=1}^{n}\left|x_{i}\right|$.\\
				令$\frac{\mathrm{d} \ln L}{\mathrm{d} \sigma}=\frac{-n}{\sigma}+\frac{1}{\sigma^{2}} \sum_{i=1}^{n}\left|x_{i}\right|=0$
				,解得$\sigma=\frac{1}{n} \sum_{i=1}^{n}\left|x_{i}\right|$,因此 $\sigma$ 的最大似然估计量为$\hat{\sigma}=\frac{1}{n} \sum_{i=1}^{n}\left|X_{i}\right|$.
				\part 因为$E(|X|)=\int_{-\infty}^{+\infty}|x| f(x) \mathrm{d} x=\int_{-\infty}^{+\infty} \frac{|x|}{2 \sigma} \mathrm{e}^{-\frac{|x|}{\sigma}} \mathrm{d} x=\sigma$
				,所以\\
			$\begin{aligned} E(\hat{\sigma}) &=\frac{1}{n} \sum_{i=1}^{n} E\left|X_{i}\right|=E(|X|)=\sigma 
			\\ E\left(X^{2}\right) &=\int_{-\infty}^{+\infty} x^{2} f(x) \mathrm{d} x=\int_{-\infty}^{+\infty} \frac{x^{2}}{2 \sigma} \mathrm{e}^{-\frac{|x|}{\sigma}} \mathrm{d} x
			\\=2 \sigma^{2} 
		\\ D(\hat{\sigma}) &=\frac{D(|X|)}{n}=\frac{1}{n}\left(E\left(X^{2}\right)-(E|X|)^{2}\right)=\frac{\sigma^{2}}{n} \end{aligned}$
			\end{parts}
		\end{solution}

	\end{questions}
\end{document}