\documentclass[kindlepaper]{BHCexam4kindle}
\renewcommand\thesection{}
\renewcommand\thesubsection{}
% \titleformat{\section}[block]{\color{blue}\Large\bfseries\filcenter}{}{1em}{}
% \titleformat{\subsection}[hang]{\bfseries\filcenter}{}{1em}{}

\usepackage{lipsum}
\begin{document}
\section{NCE4}
\clearpage
\subsection{Lesson 1
Finding Fossil man}
\par
We can read of things that happened 5,000 years ago in the Near East, where
people first learned to write. But there are some parts of the world where even
now people cannot write. The only way that they can preserve their history is to
recount it as sagas--legends handed down from one generation of story-tellers
to another. These legends are useful because they can tell us something about
migrations of people who lived long ago, but none could write down what they
did. Anthropologists wondered where the remote ancestors of the Polynesian
peoples now living in the Pacific Islands came from. The sagas of these people
explain that some of them came from Indonesia about 2,000 years ago.
But the first people who were like ourselves lived so long ago that even their
sagas, if they had any, are forgotten. So archaeologists have neither history nor
legends to help them to find out where the first 'modern men' came from.
Fortunately, however, ancient men made tools of stone, especially flint, be-
cause this is easier to shape than other kinds. They may also have used wood
and skins, but these have rotted away. Stone does not decay, and so the tools of
long ago have remained when even the bones of the men who made them have
disappeared without trace.

\clearpage
\subsection{Lesson 2
Spare that spider}
\par
Why, you may wonder, should spiders be our friends ? Because they destroy so
many insects, and insects include some of the greatest enemies of the human
race. Insects would make it impossible for us to live in the world; they would
devour all our crops and kill our flocks and herds, if it were not for the protection
we get from insect-eating animals. We owe a lot to the birds and beasts who eat
insects but all of them put together kill only a fraction of the number destroyed
by spiders. Moreover, unlike some of the other insect eaters, spiders never do
the least harm to us or our belongings.
\par
Spiders are not insects, as many people think, nor even nearly related to them.
One can tell the difference almost at a glance for a spider always has eight legs
and an insect never more than six.
\par
How many spiders are engaged in this work on our behalf ? One authority on
spiders made a census of the spiders in a grass field in the south of England, and
he estimated that there were more than 2,250,000 in one acre, that is something
like 6,000,000 spiders of different kinds on a football pitch. Spiders are busy for
at least half the year in killing insects. It is impossible to make more than the
wildest guess at how many they kill, but they are hungry creatures, not content
with only three meals a day. It has been estimated that the weight of all the in-
sects destroyed by spiders in Britain in one year would be greater than the total
weight of all the human beings in the country.
\par
T. H. GILLESPIE Spare that Spider from The Listener

\clearpage
\subsection{Lesson 3
Matterhorn man}
\par
Modern alpinists try to climb mountains by a route which will give them good
sport, and the more difficult it is, the more highly it is regarded. In the pioneering
days, however, this was not the case at all. The early climbers were looking for
the easiest way to the top because the summit was the prize they sought, especi-
ally if it had never been attained before. It is true that during their explorations
they often faced difficulties and dangers of the most perilous nature, equipped
in a manner which would make a modern climber shudder at the thought, but
they did not go out of their way to court such excitement. They had a single aim,
a solitary goal--the top!
\par
It is hard for us to realize nowadays how difficult it was for the pioneers. Ex-
cept for one or two places such as Zermatt and Chamonix, which had rapidly
become popular, Alpine villages tended to be impoverished settlements cut off
from civilization by the high mountains. Such inns as there were were generally
dirty and flea-ridden; the food simply local cheese accompanied by bread often
twelve months old, all washed down with coarse wine. Often a valley boasted no
inn at all, and climbers found shelter wherever they could--sometimes with the
local priest (who was usually as poor as his parishioners), sometimes with shep-
herds or cheesemakers. Invariably the background was the same: dirt and
poverty, and very uncomfortable. For men accustomed to eating seven-course
dinners and sleeping between fine linen sheets at home, the change to the Alps
must have been very hard indeed.
\clearpage
\subsection{Lesson 4
Seeing hands}
In the Soviet Union several cases have been reported recently of people who
can read and detect colours with their fingers, and even see through solid doors
and walls. One case concerns an 'eleven-year-old schoolgirl, Vera Petrova, who
has normal vision but who can also perceive things with different parts of her
skin, and through solid walls. This ability was first noticed by her father. One
day she came into his office and happened to put her hands on the door of a
locked safe. Suddenly she asked her father why he kept so many old newspapers
locked away there, and even described the way they were done up in bundles.
Vera's curious talent was brought to the notice of a scientific research institute
in the town of UIyanovsk, near where she lives, and in April she was given a
series of tests by a special commission of the Ministry of Health of the Russian
Federal Republic. During these tests she was able to read a newspaper through
an opaque screen and, stranger still, by moving her elbow over a child's game of
Lotto she was able to describe the figures and colours printed on it; and, in an-
other instance, wearing stockings and slippers, to make out with her foot the
outlines and colours of a picture hidden under a carpet. Other experiments
showed that her knees and shoulders had a similar sensitivity. During all these
tests Vera was blindfold; and, indeed, except when blindfold she lacked the
ability to perceive things with her skin. lt was also found that although she
could perceive things with her fingers this ability ceased the moment her hands
were wet.
\clearpage
\subsection{Lesson 5
Youth}
\par
People are always talking about' the problem of youth '. If there is one--which
I take leave to doubt--then it is older people who create it, not the young them-
selves. Let us get down to fundamentals and agree that the young are after all
human beings--people just like their elders. There is only one difference be-
tween an old man and a young one: the young man has a glorious future before
him and the old one has a splendid future behind him: and maybe that is where
the rub is.
\par
When I was a teenager, I felt that I was just young and uncertain--that I was
a new boy in a huge school, and I would have been very pleased to be regarded
as something so interesting as a problem. For one thing, being a problem gives
you a certain identity, and that is one of the things the young are busily engaged
in seeking.
\par
I find young people exciting. They have an air of freedom, and they have not a
dreary commitment to mean ambitions or love of comfort. They are not anxious
social climbers, and they have no devotion to material things. All this seems to
me to link them with life, and the origins of things. It's as if they were in some
sense cosmic beings in violent and lovely contrast with us suburban creatures.
All that is in my mind when I meet a young person. He may be conceited, ill-
mannered, presumptuous of fatuous, but I do not turn for protection to dreary
clichés about respect for elders--as if mere age were a reason for respect. I
accept that we are equals, and I will argue with him, as an equal, if I think he
is wrong.

\clearpage
\subsection{Lesson 6
The sporting spirit}
I am always amazed when I hear people saying that sport creates goodwill between
the nations, and that if only the common peoples of the world could meet
one another at football or cricket, they would have no inclination to meet on
the battlefield. Even if one didn't know from concrete examples (the 1936
Olympic Games, for instance) that international sporting contests lead to orgies
of hatred, one could deduce it from general principles.
\par
Nearly all the sports practised nowadays are competitive. You play to win,
and the game has little meaning unless you do your utmost to win. On the village
green, where you pick up sides and no feeling of local patriotism is involved, it
is possible to play simply for the fun and exercise: but as soon as the question of
prestige arises, as soon as you feel that you and some larger unit will be dis-
graced if you lose, the most savage combative instincts are aroused. Anyone who
has played even in a school football match knows this. At the international level
sport is frankly mimic warfare. But the significant thing is not the behaviour of
the players but the attitude of the spectators: and, behind the spectators, of the
nations. who work themselves into furies over these absurd contests, and seriously
believe--at any rate for short periods--that running, jumping and kicking a ball
are tests of national virtue.
\clearpage
\subsection{Lesson 7
Bats}
\par
Not all sounds made by animals serve as language, and we have only to turn to
that extraordinary discovery of echo-location in bats to see a case in which the
voice plays a strictly utilitarian role.
\par
To get a full appreciation of what this means we must turn first to some recent
human inventions. Everyone knows that if he shouts in the vicinity of a wall or
a mountainside, an echo will come back. The further off this solid obstruction
the longer time will elapse for the return of the echo. A sound made by tapping
on the hull of a ship will be reflected from the sea bottom, and by measuring the
time interval between the taps and the receipt of the echoes the depth of the
sea at that point can be calculated. So was born the echo-sounding apparatus,
now in general use in ships. Every solid object will reflect a sound, varying ac-
cording to the size and nature of the object. A shoal of fish will do this. So it is a
comparatively simple step from locating the sea bottom to locating a shoal of
fish. With experience, and with improved apparatus, it is now possible not only
to locate a shoal but to tell if it is herring, cod, or other well-known fish, by the
pattern of its echo.
\par
A few years ago it was found that certain bats emit squeaks and by receiving
the echoes they could locate and steer clear of obstacles--or locate flying insects
on which they feed. This echo-location in bats is often compared with radar, the
principle of which is similar.
\clearpage
\subsection{*Lesson 8
Trading standards}
\par
Chickens slaughtered in the United States, claim officials in Brussels, are not fit to grace European tables. No,
say the Americans: our fowl are fine, we simply clean them in a different way. These days, it is differences in
national regulations, far more than tariffs, that put sand in the wheels of trade between rich countries. It is not
just farmers who are complaining . An electric razor that meets the European Union's safety standards must be
approved by American testers before it can be sold in the United States, and an American-made dialysis machine
needs the EU's okay before it hits the market in Europe.
\par
As it happens, a razor that is safe in Europe is unlikely to electrocute Americans. So, ask businesses on both
sides of the Atlantic, why have two lots of tests where one would do? Politicians agree, in principle, so America
and the EU have been trying to reach a deal which would eliminate the need to double-test many products. They
hope to finish in time for a trade summit between America and EU on May 28th. Although negotiators are
optimistic, the details are complex enough that they may be hard-pressed to get a deal at all.
\par
Why? One difficulty is to construct the agreements. The Americans would happily reach one accord on
standards for medical devices and then hammer out different pacts covering, say, electronic goods and drug
manufacturing. The EU-following fine continental traditions—wants agreement on general principles, which
could be applied to many types of products and have extended to other countries.
\clearpage
\subsection{Lesson 9
Royal espionage}
\par
Alfred the Great acted as his own spy, visiting Danish camps disguised as a
minstrel. In those days wandering minstrels were welcome everywhere. They
were not fighting men, and their harp was their passport. Alfred had learned
many of their ballads in his youth, and could vary his programme with acrobatic
tricks and simple conjuring.
\par
While Alfred's little army slowly began to gather at Athelney, the king himself
set out to penetrate the camp of Guthrum, the commander of the Danish in-
vaders. These had settled down for the winter at Chippenham: thither Alfred
went. He noticed at once that discipline was slack: the Danes had the self-
confidence of conquerors, and their security precautions were casual. They lived
well, on the proceeds of raids on neighbouring regions. There they collected
women as well as food and drink, and a life of ease had made them soft.
Alfred stayed in the camp a week before he returned to Athelney. The force
there assembled was trivial compared with the Danish horde. But Alfred had
deduced that the Danes were no longer fit for prolonged battle : and that their
commissariat had no organization, but depended on irregular raids.
\par
So, faced with the Danish advance, Alfred did not risk open battle but harried
the enemy. He was constantly on the move, drawing the Danes after him. His
patrols halted the raiding parties: hunger assailed the Danish army. Now Alfred
began a long series of skirmishes--and within a month the Danes had surrendered. The episode could reasonably serve as a unique epic of royal espionage!
\clearpage
\subsection{*Lesson 10 Silicon valley}
\par
Technology trends may push Silicon Valley back to the future. Carver Mead, a pioneer in integrated circuits
and a professor of computer science at the California Institute of Technology, notes there are now workstations
that enable engineers to design, test and produce chips right on their desks, much the way an editor creates a
newsletter on a Macintosh. As the time and cost of making a chip drip to a few days and a few hundred dollars,
engineers may soon be free to let their imaginations soar without being penalized by expensive failures. Mead
predicts that inventors will be able to perfect powerful customized chips over a weekend at the
office—spawning a new generation of garage start-ups and giving the U.S. a jump on its foreign rivals in
getting new products to market fast. ‘We've got more garages with smart people,' Mead observes. ‘We really
thrive on anarchy.'
\par
And on Asians. Already, orientals and Asian Americans constitute the majority of the engineering staffs at
many Valley firms. And Chinese, Korean, Filipino and Indian engineers are graduating in droves from
California's colleges. As the heads of next-generation start-ups, these Asian innovators can draw on customs
and languages to forge tighter links with crucial Pacific Rim market. For instance, Alex Au, a Stanford Ph.D.
from Hong Kong, has set up a Taiwan factory to challenge Japan's near lock on the memory-chip market.
India-born N. Damodar Reddy's tiny California company reopened an AT&T chip plant in Kansas City last
spring with financing from the state of Missouri. Before it becomes a retirement village, Silicon Valley may
prove a classroom for building a global business.
\clearpage
\subsection{Lesson 11
How to grow old}
\par
Some old people are oppressed by the fear of death. In the young there is a justification for this feeling.
Young men who have reason to fear that they will be killed in battle may justifiably feel bitter in the thought
that they have been cheated of the best things that life has to offer. But in an old man who has known human
joys and sorrows, and has achieved whatever work it was in him to do, the fear of death is somewhat abject
and ignoble. The best way to overcome it-so at least it seems to me----is to make your interests gradually wider
and more impersonal, until bit by bit the walls of the ego recede, and your life becomes increasingly merged in
the universal life. An individual human existence should be like a river--small at first, narrowly contained
within its banks, and rushing passionately past boulders and over waterfalls. Gradually the river grows
wider ,the banks recede, the waters flow more quietly, and in the end, without any visible break, they become
merged in the sea, and painlessly lose their individual being. The man who, in old age, can see his life in this
way, will not suffer from the fear of death, since the things he cares for will continue. And it, with the decay of
vitality, weariness increases, the thought of rest will be not unwelcome. I should wish to die while still at work,
knowing that others will carry on what I can no longer do, and content in the thought that what was possible
has been done.
\clearpage
\subsection{Lesson 12
Banks and their customers}
\par
When anyone opens a current account at a bank, he is lending the bank money, repayment of which he
may demand at any time, either in cash or by drawing a cheque in favour of another person. Primarily, the
banker-customer relationship is that of debtor and creditor--who is which depending on whether the customer's
account is in credit or is overdrawn. But, in addition to that basically simple concept, the bank and its customer
owe a large number of obligations to one another. Many of these obligations can give rise to problems and
complications but a bank customer, unlike, say, a buyer of goods, cannot complain that the law is loaded
against him.
\par
The bank must obey its customer's instructions, and not those of anyone else. When, for example, a
customer first opens an account, he instructs the bank to debit his account only in respect of cheques drawn by
himself. He gives the bank specimens of his signature, and there is a very firm rule that the bank has no right
or authority to pay out a customer's money on a cheque on which its customer's signature has been forged.It
makes no difference that the forgery may have been a very skilful one: the bank must recognize its customer's
signature.
\par
For this reason there is no risk to the customer in the modern practice, adopted by some banks, of printing
the customer's name on his cheques. If this facilitates forgery it is the bank which will lose, not the customer.
\clearpage
\subsection{Lesson 13
The search for oil}
\par
The deepest holes of all are made for oil, and they go down to as much as 25,000
feet. But we do not need to send men down to get the oil out, as we must with
other mineral deposits. The holes are only borings, less than a foot in diameter.
\par
My particular experience is largely in oil, and the search for oil has done more to
improve deep drilling than any other mining activity. When it has been decided
where we are going to drill, we put up at the surface an oil derrick. It has to be
tall because it is like a giant block and tackle, and we have to lower into the
ground and haul out of the. ground great lengths of drill pipe which are rotated
by an engine at the top and are fitted with a cutting bit at the bottom.
\par
The geologist needs to know what rocks the drill has reached, so every so often
a sample is obtained with a coring bit. It cuts a clean cylinder of rock, from which
can be seen he strata the drill has been cutting through. Once we get down to
the oil, it usually flows to the surface because great pressure, either from gas or
water, is pushing it. This pressure must be under control, and we control it by
means of the mud which we circulate down the drill pipe. We endeavour to
avoid the old, romantic idea of a gusher, which wastes oil and gas. We want it to
stay down the hole until we can lead it off in a controlled manner.
\clearpage
\subsection{*Lesson 14
The Butterfly Effect}
\par
Beyond two or three days, the world's best weather forecasts are speculative, and beyond six or seven they
are worthless.
\par
The Butterfly Effect is the reason. For small pieces of weather—and to a global forecaster, small can mean
thunderstorms and blizzards – any prediction deteriorates rapidly. Errors and uncertainties multiply, cascading
upward through a chain of turbulent features, from dust devils and squalls up to continent-size eddies that only
satellites can see.
\par
The modern weather models work with a grid of points of the order of sixty miles apart, and even so, some
starting data has to be guessed, since ground stations and satellites cannot see everywhere. But suppose the
earth could be covered with sensors spaced one foot apart, rising at one-foot intervals all the way to to top of
the atmosphere. Suppose every sensor gives perfectly accurate readings of temperature, pressure, humidity, and
any other quantity a meteorologist would want. Precisely at noon an infinitely powerful computer takes all the
data and calculates what will happen at each point at 12.01, then 12.02, then 12.03....
\par
The computer will still be unable to predict whether Princeton, New Jersey, will have sun or rain on a day
one month away. At noon the spaces between the sensors will hide fluctuations that the computer will not
know about, tiny deviations from the average. By 1.201, those fluctuations will already have created small
errors one foot away. Soon the errors will have multiplied to the ten-foot scale, and so on up to the size of the
globe.
\clearpage
\subsection{Lesson 15
Secrecy in industry}
Two factors weigh heavily against the effectiveness of scientific in industry.
One is the general atmosphere of secrecy in which it is carried out, the
other the lack of freedom of the individual research worker. In so far as any
inquiry is a secret one, it naturally limits all those engaged in carrying it out
from effective contact with their fellow scientists either in other countries or in
universities, or even , often enough , in other departments of the same firm. The
degree of secrecy naturally varies considerably. Some of the bigger firms are engaged
in researches which are of such general and fundamental nature that it is a
positive advantage to them not to keep them secret. Yet a great many processes
depending on such research are sought for with complete secrecy until the stage
at which patents can be taken out. Even more processes are never patented at all
but kept as secret processes. This applies particularly to chemical industries,
where chance discoveries play a much larger part than they do in physical and
mechanical industries. Sometimes the secrecy goes to such an extent that the
whole nature of the research cannot be mentioned. Many firms, for instance,
have great difficulty in obtaining technical or scientific books from libraries be-
cause they are unwilling to have their names entered as having taken out such
and such a book for fear the agents of other firms should be able to trace the kind
of research they are likely to be undertaking.
\clearpage
\subsection{Lesson 16
The modern city}
In the organization of industrial life the influence of the factory upon the physiological and mental state of
the workers has been completely neglected. Modern industry is based on the conception of the maximum
production at lowest cost, in order that an individual or a group of individuals may earn as much money as
possible. It has expanded without any idea of the true nature of the human beings who run the machines, and
without giving any consideration to the effects produced on the individuals and on their descendants by the
artificial mode of existence imposed by the factory. The great cities have been built with no regard for us. The
shape and dimensions of the skyscrapers depend entirely on the necessity of obtaining the maximum income
per square foot of ground, and of offering to the tenants offices and apartments that please them. This caused
the construction of gigantic buildings where too large masses of human beings are crowded together. Civilized
men like such a way of living. While they enjoy the comfort and banal luxury of their dwelling, they do not
realize that they are deprived of the necessities of life. The modern city consists of monstrous edifices and of
dark, narrow streets full of petrol fumes, coal dust, and toxic gases, torn by the noise of the taxi-cabs, lorries
and buses, and thronged ceaselessly by great crowds. Obviously, it has no been planned for the good of its
inhabitants.
\clearpage
\subsection{Lesson 17
A man-made disease}
\par
In the early days of the settlement of Australia, enterprising settlers unwisely
introduced the European rabbit. This rabbit had no natural enemies in the An-
tipodes, so that it multiplied with that promiscuous abandon characteristic of
rabbits. It overran a whole continent. It caused devastation by burrowing and
by devouring the herbage which might have maintained millions of sheep and
cattle. Scientists discovered that this particular variety of rabbit (and apparently
no other animal) was susceptible to a fatal virus disease, myxomatosis. By infect-
ing animals and letting them loose in the burrows, local epidemics of this disease
could be created. Later it was found that there was a type of mosquito which
acted as the carrier of this disease and passed it on to the rabbits. So while the
rest of the world was trying to get rid of mosquitoes, Australia was encouraging
this one. It effectively spread the disease all over the continent and drastically
reduced the rabbit population. lt later became apparent that rabbits were de-
veloping a degree of resistance to this disease, so that the rabbit population was
unlikely to be completely exterminated. There were hopes, however, that the
problem of the rabbit would become manageable.
\par
Ironically, Europe, which had bequeathed the rabbit as a pest to Australia
acquired this man-made disease as a pestilence. A French physician decided to
get rid of the wild rabbits on his own estate and introduced myxomatosis. It did
not, however, remain within the confines of his estate. It spread through France
where wild rabbits are not generally regarded as a pest but as a sport and a useful
food supply, and it spread to Britain where wild rabbits are regarded as a pest
but where domesticated rabbits, equally susceptible to the disease, are the basis
of a profitable fur industry. The question became one of whether Man could con-
trol the disease he had invented.
\clearpage
\subsection{Lesson 18
Porpoises}
\par
There has long been a superstition among mariners that porpoises will save
drowning men by pushing them to the surface, or protect them from sharks by
surrounding them in defensive formation. Marine Studio biologists have pointed
out that, however intelligent they may be, it is probably a mistake to credit dol-
phins with any motive of life-saving. On the occasions when they have pushed to
shore an unconscious human being they have much more likely done it out of
curiosity or for sport,as in riding the bow waves of a ship. In 1928 some porpoises
were photographed working like beavers to push ashore a waterlogged mattress.
If, as has been reported, they have protected humans from sharks, it may have
been because curiosity attracted them and because the scent of a possible meal
attracted the sharks. Porpoises and sharks are natural enemies. It is possible
that upon such an occasion a battle ensued, with the sharks being driven away
or killed.
\par
Whether it be bird, fish or beast, the porpoise is intrigued with anything that
is alive. They are constantly after the turtles, the Ferdinands of marine life, who
peacefully submit to all sorts of indignities. One young calf especially enjoyed
raising a turtle to the surface with his snout and then shoving him across the
tank like an aquaplane. Almost any day a young porpoise may be seen trying
to turn a 300-pound sea turtle over by sticking his snout under the edge of his
shell and pushing up for dear life. This is not easy, and may require two porpoises
working together. In another game, as the turtle swims across the oceanarium,
the first porpoise swoops down from above and butts his shell with his belly.
This knocks the turtle down several feet. He no sooner recovers his equilibrium
than the next porpoise comes along and hits him another crack. Eventually the
turtle has been butted all the way down to the floor of the tank. He is now satis-
fied merely to try to stand up, but as soon as he does so a porpoise knocks him
flat. The turtle at last gives up by pulling his feet under his shell and the game
is over.
\clearpage
\subsection{Lesson 19
The stuff of dreams}
\par
It is fairly clear that the sleeping period must have some function, and because
there is so much of it the function would seem to be important. Speculations
about its nature have been going on for literally thousands of years, and one odd
finding that makes the problem puzzling is that it looks very much as if sleeping
is not simply a matter of giving the body a rest.' Rest ', in terms of muscle relaxa-
tion and so on, can be achieved by a brief period lying, or even sitting down. The
body's tissues are self-repairing and self-restoring to a degree, and function best
when more or less continuously active. In fact a basic amount of movement occurs
during sleep which is specifically concerned with preventing muscle inactivity.
\par
If it is not a question of resting the body, then perhaps it is the brain that needs
resting? This might be a plausible hypothesis were it not for two factors. First the
electroencephalograph (which is simply a device for recording the electrical
activity of the brain by attaching electrodes to the scalp) shows that while there
is a change in the pattern of activity during sleep, there is no evidence that the
total amount of activity is any less. The second factor is more interesting and
more fundamental. In l960 an American psychiatrist named William Dement
published experiments dealing with the recording of eye-movements during
sleep. He showed that the average individual's sleep cycle is punctuated with
peculiar bursts of eye-movements, some drifting and slow, others jerky and rapid.
\par
People woken during these periods of eye-movements generally reported that
they had been dreaming. When woken at other times they reported no dreams. If
one group of people were disturbed from their eye-movement sleep for several
nights on end, and another group were disturbed for an equal period of time but
when they were not exhibiting eye-movements, the first group began to show
some personality disorders while the others seemed more or less unaffected. The
implications of all this were that it was not the disturbance of sleep that mattered,
but the disturbance of dreaming.
\clearpage
\subsection{Lesson 20
Snake poison}
\par
How it came about that snakes manufactured poison is a mystery. Over the
periods their saliva, a mild, digestive juice like our own, was converted into a
poison that defies analysis even today. It was not forced upon them by the sur-
vival competition; they could have caught and lived on prey without using
poison just as the thousands of non-poisonous snakes still do. Poison to a snake
is merely a luxury; it enables it to get its food with very little effort, no more
effort than one bite. And why only snakes ? Cats, for instance, would be greatly
helped; no running rights with large, fierce rats or tussles with grown rabbits-
just a bite and no more effort needed. In fact it would be an assistance to all the
carnivorae--though it would be a two-edged weapon -When they fought each
other. But, of the vertebrates, unpredictable Nature selected only snakes (and
one lizard). One wonders also why Nature, with some snakes concocted poison
of such extreme potency.
\par
In the conversion of saliva into poison one might suppose that a fixed process
took place. It did not; some snakes manufactured a poison different in every re-
spect from that of others, as different as arsenic is from strychnine, and having
different effects. One poison acts on the nerves, the other on the blood.
\par
The makers of the nerve poison include the mambas and the cobras and their
venom is called neurotoxic. Vipers (adders) and rattlesnakes manufacture the
blood poison, which is known as haemolytic. Both poisons are unpleasant, but
by far the more unpleasant is the blood poison. It is said that the nerve poison
is the more primitive of the two, that the blood poison is , so to speak, a newer
product from an improved formula. Be that as it may, the nerve poison does its
business with man far more quickly than the blood poison. This,however,means
nothing. Snakes did not acquire their poison for use against man but for use
against prey such as rats and mice, and the effects on these of viperine poison is
almost immediate.
\clearpage
\subsection{Lesson 21
William S. Hart and the early ‘Western' film}
\par
William S. Hart was, perhaps, the greatest of all Western stars, for unlike Gary
Cooper and John Wayne he appeared in nothing but Westerns. From 1914 to
1924 he was supreme and unchallenged. It was Hart who created the basic
formula of the Western film, and devised the protagonist he played in every film
he made, the good-bad man, the accidental, noble outlaw, or the honest but
framed cowboy, or the sheriff made suspect by vicious gossip; in short, the indi-
vidual in conflict with himself and his frontier environment.
Unlike most of his contemporaries in Hollywood, Hart actually 'knew some-
thing of the old West. He had lived in it as a child when it was already disappear-
ing, and his hero was firmly rooted in his memories and experiences, and in both
the history and the mythology of the vanished frontier. And although no period
or place in American history has been more absurdly romanticized, myth and
reality did join hands in at least one arena, the conflict between the individual
and encroaching civilization.
\par
Men accustomed to struggling for survival against the elements and Indian
were bewildered by politicians, bankers and business-men, and unhorsed by
fences, laws and alien taboos. Hart's good-bad man was always an outsider,
always one of the disinherited, and if he found it necessary to shoot a sheriff or
rob a bank along the way, his early audiences found it easy to understand and
forgive, especially when it was Hart who, in the end, overcame the attacking
Indians.
\par
Audiences in the second decade of the twentieth century found it pleasant to
escape to a time when life, though hard, was relatively simple. We still do; living
in a world in which undeclared aggression, war, hypocrisy, chicanery, anarchy
and impending immolation are part of our daily lives, we all want a code to
live by.
\clearpage
\subsection{Lesson 22
Knowledge and progress}
\par
Why does the idea of progress loom so large in the modern world ? Surely be-
cause progress of a particular kind is actually taking place around us and is
becoming more and more manifest. Although mankind has undergone no general
improvement in intelligence or morality, it has made extraordinary progress
the accumulation of knowledge. Knowledge began to increase as soon as the
thoughts of one individual could be communicated to another by means of
speech. With the invention of writing, a great advance was made, for knowledge
could then be not only communicated but also stored. Libraries made education
possible, and education in its turn added to libraries: the growth of knowledge
followed a kind of compound-interest law, which was greatly enhanced by the
invention of printing. All this was comparatively slow until, with the coming
science, the tempo was suddenly raised. Then knowledge began to be accumu-
lated according to a systematic plan. The trickle became a stream; the stream
has now become a torrent. Moreover, as soon as new knowledge is acquired, it
is now turned to practical account. What is called 'modern civilization' is not
the result of a balanced development of all man's nature, but of accumulated
knowledge applied to practical life. The problem now facing humanity is: What
is going to be done with all this knowledge ? As is so often pointed out, knowledge
is a two-edged weapon which can be used equally for good or evil. It is now being
used indifferently for both. Could any spectacle, for instance, be more grimly
whimsical than that of gunners using science to shatter men's bodies while, close
at hand, surgeons use it to restore them ? We have to ask ourselves very seriously
what will happen if this twofold use of knowledge, with its ever-increasing
power, continues.
\clearpage
\subsection{Lesson 23
Bird flight}
\par
No two sorts of birds practise quite the same sort of flight; the varieties are infi-
nite, but two classes may be roughly seen. Any ship that crosses the pacific is
accompanied for many days by the smaller albatross, which may keep company
with the vessel for an hour without visible or more than occasional movement of
wing. The currents of air that the walls of the ship direct upwards, as well as in
the line of its course are enough to give the great bird with its immense wings
sufficient sustenance and progress. The albatross is the king of the gliders, the
class of fliers which harness the air to their purpose, but must yield to its opposi-
tion. In the contrary school the duck is supreme. It comes nearer to the engines
with which man has 'conquered' the air, as he boasts. Duck, and like them the
pigeons, are endowed with steel-like muscles, that are a good part of the weight
of the bird, and these will ply the short wings with irresistible power that they
can bore for long distances through an opposite gale before exhaustion follows.
Their humbler followers, such as partridges, have a like power of strong propul-
sion, but soon tire. You may pick them up in utter exhaustion, if wind over the
sea has driven them to a long journey. The swallow shares the virtues of both
schools in highest measure. It tires not nor does it boast of its power; but belongs
to the air, travelling it may be six thousand miles to and from its northern nesting
home feeding its flown young as it flies and slipping through a medium that
seems to help its passage even when the wind is adverse. Such birds do us good,
though we no longer take omens from their flight on this side and that, and even
the most superstitious villagers no longer take off their hats to the magpie and
wish it good-morning.
\clearpage
\subsection{Lesson 24
Beauty}
\par
A young man sees a sunset and, unable to understand or to express the emotion
that it rouses in him, concludes that it must be the gateway to a world that lies
beyond. It is difficult for any of us in moments of intense aesthetic experience to
resist the suggestion that we are catching a glimpse of a light that shines down
to us from a different realm of existence, different and, because the experience is
intensely moving, in some way higher. And, though the gleams blind and dazzle,
yet do they convey a hint of beauty and serenity greater than we have known or
imagined. Greater too than we can describe, for language, which was invented
to convey the meanings of this world, cannot readily be fitted to the uses of
another.
\par
That all great art has this power of suggesting a world beyond is undeniable.
In some moods Nature shares it. There is no sky in June so blue that it does not
point forward to a bluer, no sunset so beautiful that it does not waken the vision
of a greater beauty, a vision which passes before it is fully glimpsed, and in
passing leaves an indefinable longing and regret. But, if this world is not merely
a bad joke, life a vulgar flare amid the cool radiance of the stars, and existence
an empty laugh braying across the mysteries; if these intimations of a something
behind and beyond are not evil humour born of indigestion, or whimsies sent by
the devil to mock and madden us, if, in a word, beauty means something, yet we
must not seek to interpret the meaning. If we glimpse the unutterable, it is un-
wise to try to utter it, nor should we seek to invest with significance that which
we cannot grasp. Beauty in terms of our human meanings is meaningless.
\clearpage
\subsection{Lesson 25
Non-auditory effects of noise}
\par
Many people in industry and the Services, who have practical experience of
noise, regard any investigation of this question as a waste of time; they are not
prepared even to admit the possibility that noise affects people. On the other
hand, those who dislike noise will sometimes use most inadequate evidence to
support their pleas for a quieter society. This is a pity, because noise abatement
really is a good cause. and it is likely to be discredited if it gets to be associated
with bad science.
\par
One allegation often made is that noise produces mental illness. A recent article
in a weekly newspaper, for instance, was headed with a striking illustration of a
lady in a state of considerable distress, with the caption 'She was yet another
victim, reduced to a screaming wreck '. On turning eagerly to the text, one learns
that the lady was a typist who found the sound of office typewriters worried her
more and more until eventually she had to go into a mental hospital. Now the
snag in this sort of anecdote is of course that one cannot distinguish cause and
effect. Was the noise a cause of the illness, or were the complaints about noise
merely a symptom? Another patient might equally well complain that her neigh-
bours were combining to slander her and persecute her, and yet one might be
cautious about believing this statement.
\par
What is needed in the case of noise is a study of large numbers of people living
under noisy conditions, to discover whether they are mentally ill more often than
other people are. The United States Navy, for instance, recently examined a very
large number of men working on aircraft carriers: the study was known as
Project Anehin. It can be unpleasant to live even several miles from an aerodrome;
if you think what it must be like to share the deck of a ship with several squad-
rons of jet aircraft, you will realize that a modern navy is a good place to study
noise. But neither psychiatric interviews nor objective tests were able to show
any effects upon these American sailors. This result merely confirms earlier
American and British studies: if there is any effect of noise upon mental health
it must be so small that present methods of psychiatric diagnosis cannot find it.
That does not prove that it does not exist; but it does mean that noise is less
dangerous than, say, being brought up in an orphanages--which really is a mental
health hazard.
\clearpage
\subsection{Lesson 26
The past life of the earth}
\par
It is animals and plants which lived in or near water whose remains are most
likely to be preserved, for one of the necessary conditions of preservation is quick
burial, and it is only in the seas and rivers, and sometimes lakes, where mud and
silt have been continuously deposited, that bodies and the like can be rapidly
covered over and preserved.
\par
But even in the most favourable circumstances only a small fraction of the
creatures that die are preserved in this way before decay sets in or, even more
likely, before scavengers eat them. After all, all living creatures live by feeding
on something else, whether it be plant or animal, dead or alive, and it is only by
chance that such a fate is avoided. The remains of plants and animals that lived
on land are much more rarely preserved, for there is seldom anything to cover
them over. When you think of the innumerable birds that one sees flying about,
not to mention the equally numerous small animals like field mice and voles
which you do not see, it is very rarely that one comes across a dead body, except,
of course, on the roads. They decompose and are quickly destroyed by the
weather or eaten by some other creature.
\par
It is almost always due to some very special circumstances that traces of land
animals survive, as by falling into inaccessible caves, or into an ice crevasse, like
the Siberian mammoths, when the whole animal is sometimes preserved, as in
a refrigerator. This is what happened to the famous Beresovka mammoth which
was found preserved and in good condition. In his mouth were the remains of
fir trees--the last meal that he had before he fell into the crevasse and broke his
back. The mammoth has now been restored in the Palaeontological Museum in
Leningrad. Other animals were trapped in tar pits, like the elephants, sabre-
toothed cats, and numerous other creatures that are found at Rancho la Brea,
which is now just a suburb of Los Angeles. Apparently what happened was that
water collected on these tar pits, and the bigger animals like the elephants ven-
tured out on to the apparently firm surface to drink, and were promptly bogged
in the tar. And then, when they were dead, the carnivores, like the sabre-toothed
cats and the giant wolves, came out to feed and suffered exactly the same fate.
There are also endless numbers of birds in the tar as well.
\clearpage
\subsection{Lesson 27
The ‘Vasa'}
\par
From the seventeenth-century empire of Sweden, the story of a galleon that
sank at the start of her maiden voyage in 1628 must be one of the strangest tales
of the sea. For nearly three and a half centuries she lay at the bottom of Stock-
holm harbour until her discovery in 1956. This was the Vasa, royal flagship of
the great imperial fleet.
\par
King Gustavus Adolphus, 'The Northern Hurricane', then at the height of
his military success in the Thirty Years' War, had dictated her measurements
and armament. Triple gun-decks mounted sixty-four bronze cannon. She was
intended to play a leading role in the growing might of Sweden.
As she was prepared for her maiden voyage on August 10, 1628, Stockholm
was in a ferment. From the Skeppsbron and surrounding islands the people
watched this thing of beauty begin to spread her sails and catch the wind. They
had laboured for three years to produce this floating work of art; she was more
richly carved and ornamented than any previous ship. The high stern castle was
a riot of carved gods, demons, knights, kings, warriors,mermaids, cherubs; and
zoomorphic animal shapes ablaze with red and gold and blue, symbols of courage,
power, and cruelty, were portrayed to stir the imaginations of the superstitious
sailors of the day.
\par
Then the cannons of the anchored warships thundered a salute to which the
Vasa fired in reply. As she emerged from her drifting cloud of gun smoke with
the water churned to foam beneath her bow, her flags flying, pennants waving,
sails filling in the breeze, and the red and gold of her superstructure ablaze with
colour, she presented a more majestic spectacle than Stockholmers had ever seen
before. All gun-ports were open and the muzzles peeped wickedly from them.
\par
As the wind freshened there came a sudden squall and the ship made a strange
movement, listing to port. The Ordnance Officer ordered all the port cannon to
be heaved to starboard to counteract the list, but the steepening angle of the decks
increased. Then the sound of rumbling thunder reached the watchers on the
shore, as cargo, ballast, ammunition and 400 people went sliding and crashing
down to the port side of the steeply listing ship. The lower gun-ports were now
below water and the inrush sealed the ship's fate. In that first glorious hour, the
mighty Vasa, which was intended to rule the Baltic, sank with all flags flying--in
the harbour of her birth.
\clearpage
\subsection{Lesson 28
Patients and doctors}
\par
This is a sceptical age, but although our faith in many of the things in which our
forefathers fervently believed has weakened, our confidence in the curative
properties of the bottle of medicine remains the same as theirs. This modern
faith in medicines is roved by the fact that the annual drug bill of the Health
Services is mounting to astronomical figures and shows no signs at present of
ceasing to rise. The majority of the patients attending the medical out-patients
departments of our hospitals feel that they have not received adequate treatment
unless they are able to carry home with them some tangible remedy in the shape
of a bottle of medicine, a box of pills, or a small jar of ointment, and the doctor
in charge of the department is only too ready to provide them with these require-
ments. There is no quicker method of disposing of patients than by giving them
what they are asking for, and since most medical men in the Health Services are
overworked and have little time for offering time-consuming and little-appre-
ciated advice on such subjects as diet, right living, and the need for abandoning
bad habits, etc., the bottle, the box, and the jar are almost always granted them.
Nor is it only the ignorant and ill-educated person who has such faith in the
bottle of medicine, especially if it be wrapped in white paper and sealed with a
dab of red sealing-wax by a clever chemist. It is recounted of Thomas Carlyle
that when he heard of the illness of his friend, Henry Taylor, he went off
immediately to visit him, carrying with him in his pocket what remained of a
bottle of medicine formerly prescribed for an indisposition of Mrs Carlyle's.
Carlyle was entirely ignorant of what the bottle in his pocket contained, of the
nature of the illness from which his friend was suffering, and of what had pre-
viously been wrong with his wife, but a medicine that had worked so well in one
form of illness would surely be of equal benefit in another, and comforted by
the thought of the help he was bringing to his friend, he hastened to Henry
Taylor's house. History does not relate whether his friend accepted his medical
help, but in all probability he did. The great advantage of taking medicine is that
it makes no demands on the taker beyond that of putting up for a moment with a
disgusting taste, and that is what all patients demand of their doctors-- to be
cured at no inconvenience to themselves.
\clearpage
\subsection{Lesson 29
The hovercraft}
\par
Many strange new means of transport have been developed in our century, the
strangest of them being perhaps the hovercraft. In 1953, a former electronics
engineer in his fifties, Christopher Cockerell, who had turned to boat-building
on the Norfolk Broads, suggested an idea on which he had been working for
many years to the British Government and industrial circles. It was the idea of
supporting a craft on a' pad ', or cushion, of low-pressure air, ringed with a cur-
tain of higher pressure air. Ever since, people have had difficulty in deciding
whether the craft should be ranged among ships, planes, or land vehicles--for it
is something in between a boat and an aircraft. As a shipbuilder, Cockerell was
trying to find a solution to the problem of the wave resistance which wastes a good
deal of a surface ship's power and limits its speed. His answer was to lift the
vessel out of the water by making it ride on a cushion of air, no more than one or
two feet thick. This is done by a great number of ring-shaped air jets on the
bottom of the craft. It 'flies', therefore, but it cannot fly higher--its action de-
pends on the surface, water or ground, over which it rides.
\par
The first tests on the Solent in 1959 caused a sensation. The hovercraft
travelled first over the water, then mounted the beach, climbed up the dunes,
and sat down on a road. Later it crossed the Channel, riding smoothly over the
waves, which presented no problem.
\par
Since that time, various types of hovercraft have appeared and taken up regular
service--cruises on the Thames in London, for instance, have become an annual
attraction. But we are only at the beginning of a development that may transport net-
sea and land transport. Christopher Cockerell's craft can establish transport
works in large areas with poor communications such as Africa or Australia; it
can become a 'flying fruit-bowl', carrying bananas from the plantations to the
ports, giant hovercraft liners could span the Atlantic; and the railway of the
future may well be the 'hovertrain', riding on its air cushion over a single rail,
which it never touches, at speeds up to 300 m.p.h.--the possibilities appear
unlimited.
\clearpage

\subsection{Lesson 30
Exploring the sea-floor}
\par
Our knowledge of the oceans a hundred years ago was confined to the two-dimen-
sional shape of the sea-surface and the hazards of navigation presented by the
irregularities in depth of the shallow water close to the land. The open sea was
deep and mysterious,and anyone who gave more than a passing thought to the
bottom confines of the oceans probably assumed that the sea-bed was flat. Sir
James Clark Ross had obtained a sounding of over 2,400 fathoms in 1836 but
it was not until 1800, when H.M.S. Porcupine was put at the disposal of the
Royal Society for several cruises, that a series of deep soundings was obtained
in the Atlantic and the first samples were collected by dredging the bottom.
\par
Shortly after this the famous H.M.S. Challenger expedition established the study
of the sea-floor as a subject worthy of the most qualified physicists and geologists.
A burst of activity associated with the laying of submarine cables soon confirmed
the Challenger's observation that many parts of the ocean were two to three miles
deep, and the existence of underwater features of considerable magnitude.
\par
Today enough soundings are available to enable a relief map of the Atlantic to
be drawn and we know something of the great variety of the sea-bed's topo-
graphy. Since the sea covers the greater part of the earth's surface it is quite
reasonable to regard the sea-floor as the basic form of the crust of the earth, with
superimposed upon it the continents, together with the islands and other features
of the oceans. The continents form rugged tablelands which stand nearly three
miles above the floor of the open ocean. From the shore-line out to a distance
which may be anywhere from a few miles to a few hundred miles runs the gentle
slope of the continental shelf, geologically part of the continents. The real
dividing-line between continents and oceans occurs at the foot of a steeper slope.
\par
This continental slope usually starts at a place somewhere near the ice-fathom
mark and in the course of a few hundred miles reaches the true ocean-floor at
2,500-3,000 fathoms. The slope averages about 1 in 30, but contains steep,
probably vertical, cliffs, and gentle sediment-covered terraces, and near its lower
reaches there is a long tailing-off which is almost certainly the result of material
transported out to deep water after being eroded from the continental masses.
\clearpage
\subsection{Lesson 31
The sculptor speaks}
\par
Appreciation of sculpture depends upon the abi8lity to respond to form in three
dimensions. That is perhaps why sculpture has been described as the most
difficult of all arts; certainly it is more difficult than the arts which involve ap-
preciation of flat forms, shape in only two dimensions. Many more people are
'form-blind' than colour-blind. The child learning to see, first distinguishes only
two-dimensional shape; it cannot judge distances,depths. Later, for its personal
safety and practical needs, it has to develop(partly by means of touch) the ability
to judge roughly three-dimensional distances. But having satisfied the require-
ments of practical necessity, most people go no further. Though they may attain
considerable accuracy in the perception of flat form, they do not make the further
intellectual and emotional effort needed to comprehend form in its full spatial
existence.
\par
this is what the sculptor must do. He must strive continually to think of , and
use, form in its full spatial completeness. He gets the solid shape, as it were, in-
side his head--he thinks of it, whatever its size, as if he were holding it completely
enclosed in the hollow of his hand. He mentally visualizes a complex form from
all round itself; he knows while he looks at one side what the other side is like;
he identifies himself with its centre of gravity, its mass, its weight; he realizes
its volume, as the space that the shape displaces in the air.
\par
And the sensitive observer of sculpture must also learn to feel shape simply as
shape, not as description or reminiscence. He must, for example, perceive an
egg as a simple single solid shape, quite apart from its significance as food,or
from the literary idea that it will become a bird. And so with solids such as a
shell, a nut, a plum, a pear, a tadpole, a mushroom, a mountain peak, a kidney, a
carrot, a tree-trunk, a bird, a bud, a lark, a ladybird, a bulrush, a bone. From
these he can go on to appreciate more complex forms of combinations of several
forms.
\clearpage
\subsection{Lesson 32
Galileo reborn}
\par
In his own lifetime Galileo was the centre of violent controversy; but the scien-
tific dust has long since settled, and today we can see even his famous clash with
the Inquisition in something like its proper perspective. But, in contrast, it is only
in modern times that Galileo has become a problem child for historians of
science.
\par
The old view of Galileo was delightfully uncomplicated. He was, above all, a
man who experimented: who despised the prejudices and book learning of the
Aristotelians, who put his questions to nature instead of to the ancients, and who
drew his conclusions fearlessly. He had been the first to turn a telescope to the
sky, and he had seen there evidence enough to overthrow Aristotle and Ptolemy
together. He was the man who climbed the Leaning Tower of Pisa and dropped
various weights from the top, who rolled balls down inclined planes, and then
generalized the results of his many experiments into the famous law of free fall.
\par
But a closer study of the evidence, supported by a deeper sense of the period,
and particularly by a new consciousness of the philosophical undercurrents in
the scientific revolution, has profoundly modified this view of Galileo. Today,
although the old Galileo lives on in many popular writings, among historians of
science a new and more sophisticated picture has emerged. At the same time our
sympathy for Balileo's opponents has grown somewhat. His telescopic observa-
tion are justly immortal; they aroused great interest at the time, they had im-
portant theoretical consequences, and they provided a striking demonstration of
the potentialities hidden in instruments and apparatus. But can we blame those
who looked and failed to see what Galileo saw, if we remember that to use a
telescope at the limit of its powers calls for long experience and intimate famili-
arity with one's instrument? Was the philosopher who refused to look through
Galileo's telescope more culpable than those who alleged that the spiral nebulae
observed with Lord Rosse's great telescope in the eighteen-forties were scratches
left by the grinder? We can perhaps forgive those who said the moons of Jupiter
were produced by Galileo's spy-glass if we recall that in his day, as for centuries
before, curved glass was the popular contrivance for producing not truth but
illusion, untruth; and if a single curved glass would distort nature, how much
more would a pair of them?
\clearpage
\subsection{Lesson 33
Education}
\par
Education is one of the key words of our time. A man without an education,
many of us believe, is an unfortunate victim of adverse circumstances deprived
of one of the greatest twentieth-century opportunities. Convinced of the im-
portance of education, modern states 'invest' in institutions of learning to get
back 'interest' in the form of a large group of enlightened young men and women
who are potential leaders. Education, with its cycles of instruction so carefully
worked out, punctuated by text-books--those purchasable wells of wisdom--
what would civilization be like without its benefits ?
\par
So much is certain: that we would have doctors and preachers, lawyers and
defendantS, marriages and births--but our spiritual outlook would be different.
We would lay less stress on 'facts and figures' and more on a good memory, on
applied psychology, and on the capacity of a man to get along with his fellow-
citizens. If our educational system were fashioned after its bookless past we
would have the most democratic form of 'college' imaginable. Among the people
whom we like to call savages all knowledge inherited by tradition is shared by
all; it is taught to every member of the tribe so that in this respect everybody is,
equally equipped for life.
\par
It is the ideal condition of the 'equal start' which only our most progressive
forms of modern education try to regain. In primitive cultures the obligation to
seek and to receive the traditional instruction is binding to all. There are no
'illiterates '--if the term can be applied to peoples without a script--while our
own compulsory school attendance became law in Germany in 1642, in France
in 1806, and in England in 1876, and is still non-existent in a number of 'civi-
lized' nations. This shows how long it was before we deemed it necessary to
make sure that all our children could share in the knowledge accumulated by the
'happy few' during the past centuries.
\par
Education in the wilderness is not a matter of monetary means. All are entitled
to an equal start. There is none of the hurry which, in our society, often hampers
the full development of a growing personality. There, a child grows up under
the ever-present attention of his parents, therefore the jungles and the savannahs
know of no 'juvenile delinquency.' No necessity of making a living away from
home results in neglect of children, and no father is confronted with his inability
to 'buy' an education for his child.
\clearpage
\subsection{Lesson 34
Adolescence}
\par
Parents are often upset when their children praise the homes of their friends and
regard it as a slur on their own cooking, or cleaning, or furniture, and often are
foolish enough to let the adolescents see that they are annoyed. They may even
accuse them of disloyalty, or make some spiteful remark about the friends'
parents. Such a loss of dignity and descent into childish behaviour on the part of
the adults deeply shocks the adolescents, and makes them resolve that in future
they will not talk to their parents about the places or people they visit. Before
very long the parents will be complaining that the child is so secretive and never
tells them anything, but they seldom realize that they have brought this on
themselves.
\par
Disillusionment with the parents, however good and adequate they may be
both as parents and as individuals, is to some degree inevitable. Most children
have such a high ideal of their parents, unless the parents themselves have been
unsatisfactory, that it ca hardly hope to stand up to a realistic evaluation. Parents
would be greatly surprised and deeply touched if they realize how much belief
their children usually have in their character and infallibility, and how much this
faith means to a child. If parents were prepared for this adolescent reaction, and
realized that it was a sign that the child was growing up and developing valuable
powers of observation and independent judgement, they would not be so hurt,
and therefore would not drive the child into opposition by resenting and resist-
ing it.
\par
The adolescent, with his passion for sincerity,always respects a parent who
admits that he is wrong, or ignorant, or even that he has been unfair or unjust.
What the child cannot forgive is the parents' refusal to admit these charges if the
child knows them to be true.
\par
Victorian parents believed that they kept their dignity by retreating behind an
unreasoning authoritarian attitude; in fact hey did nothing of the kind, but
children were then too cowed to let them know how they really felt. Today we
tend to go to the other extreme, but on the whole this is a healthier attitude both
for the child and the parent. It is always wiser and safer to face up to reality,
however painful it may be at the moment.
\clearpage
\subsection{*Lesson 35 Space odyssey}
\par
The Moon is likely to become the industrial hub of the Solar System, supplying the rocket fuels for its ships,
easily obtainable from the lunar rocks in the form of liquid oxygen. The reason lies in its gravity. Because the
Moon has only an eightieth of the Earth's mass, it requires 97 percent less energy to travel the quarter of a
million miles from the Moon to Earth-orbit than the 200 mile-journey from Earth's surface into orbit!
This may sound fantastic, but it is easily calculated. To escape from the Earth in a rocket, one must travel
at seven miles per second. The comparable speed from the Moon is only 1.5 miles per second. Because the
gravity on the Moon's surface is only a sixth of Earth's (remember how easily the Apollo astronauts bounded
along), it takes much less energy to accelerate to that 1.5 miles per second than it does on Earth.
Moon-dwellers will be able to fly in space at only three percent of the cost of similar journeys by their
terrestrial cousins.
\par
Arthur C.Clark once suggested a revolutionary idea passes through three phases:
\\1. ‘It's impossible – don't waste my time.'
\\2. ‘It's possible, but not worth doing.'
\\3. ‘I said it was a god idea all along.'
\par
The idea of colonizing Mars – a world 160 times more distant than the Moon – will move decisively from
the second phase to the third, when a significant number of people are living permanently in space. Mars
has an extraordinary fascination for would –by voyagers. America, Russia and Europe are filled with
enthusiasts – many of them serious and senior scientists – who dream of sending people to it. Their aim is
understandable. It is the one world in the Solar System that is most like the Earth. It is a world of red
sandy deserts( hence its name – the Red Planet), cloudless skies, savage sandstorms, chasms wider than
the Grand Canyon and at least one mountain more than twice as tall as Everest. It seems ideal for
settlement.
\clearpage
\subsection{Lesson 36
The cost of government}
\par
If a nation is essentially disunited, it is left to the government to hold it together.
This increases the expense of government, and reduces correspondingly the
amount of economic resources that could be used for developing the country,
And it should not be forgotten how small those resources are in a poor and back-
ward country. Where the cost of government is high, resources for development
are correspondingly low.
\par
This may be illustrated by comparing the position of a nation with that of a
private business enterprise. An enterprise has to incur certain costs and expenses
in order to stay in business. For our purposes, we are concerned only with one
kind of cost--the cost of managing and administering the business. Such adminis-
trative overhead in a business is analogous to the cost of government in a nation.
\par
The administrative overhead of a business is low to the extent that everyone
working in the business can, be trusted to behave in a way that best promotes the
interests of the firm. If they can each be trusted to take such responsibilities,
and to exercise such initiative as falls within their sphere, then administrative
overhead will be low. It will be low because it will be necessary to have only one
man looking after each job, without having another man to check upon what he
is doing, keep him in line, and report on him to someone else. But if no one can
be trusted to act in a loyal and responsible manner towards his job, then the
business will require armies of administrators, checkers, and foremen, and ad-
ministrative overhead will rise correspondingly. As administrative overhead rises,
so the earnings of the business, after meeting the expense of administration, will
fall; and the business will have less money to distribute as dividends or invest
directly in its future progress and development.
\par
It is precisely the same with a nation. To the extent that the people can be
relied upon to behave in a loyal and responsible manner, the government does
not require armies of police and civil servants to keep them in order. But if a
nation is disunited, the government cannot be sure that the actions of the people
will be in the interests of the nation; and it will have to watch, check, and control
the people accordingly. A disunited nation therefore has to incur unduly high
costs of government.
\clearpage
\subsection{Lesson 37
The process of ageing}
\par
At the age of twelve years, the human body is at its most vigorous. It has yet to
reach its full size and strength, and its owner his or her full intelligence; but at
this age the likelihood of death is least. Earlier we were infants and young child-
ren, and consequently more vulnerable; later, we shall undergo a progressive loss
of our vigour and resistance which, though imperceptible at first, will finally be-
come so steep that we can live no longer, however well we look after ourselves,
and however well society, and our doctors, look after us. This decline in vigour
with the passing of time is called ageing. It is one of the most unpleasant dis-
coveries which we all make that we must decline in this way, that if we escape
wars, accidents and diseases we shall eventually die of old age, and that this
happens at a rate which differs little from person to person, so that there are
heavy odds in favour of our dying between the ages of sixty-five and eighty. Some
of us will die sooner, a few will live longer-- on into a ninth or tenth decade. But
the chances are against it, and there is a virtual limit on how long we can hope
to remain alive, however lucky and robust we are.
\par
Normal people tend to forget this process unless and until they are reminded
of it. We are so familiar with the fact that man ages, that people have for years
assumed that the process of losing vigour with time, of becoming more likely to
die the older we get, was something self-evident, like the cooling of a hot kettle
or the wearing-out of a pair of shoes. They have also assumed that all animals,
and probably other organisms such as trees, or even the universe itself, must in
the nature of things 'wear out'. Most animals we commonly observe do in fact
age as we do if given the chance to live long enough; and mechanical systems like
a wound watch or the sun, do in fact run out of energy in accordance with the
second law of thermodynamics (whether the whole universe does so is a moot
point at present). But these are not analogous to what happens when man ages
A run-down watch is still a watch and can be rewound. An old watch, by con-
trast, becomes so worn and unreliable that it eventually is not worth mending.
\par
But a watch could never repair itself it does not consist of living parts, only of
metal, which wears away by friction. We could,at one time, repair ourselves
well enough, at least, to overcome all but the most instantly fatal illnesses an
accidents. Between twelve and eighty years we gradually lose this power; an
illness which at twelve would knock us over, at eighty can knock us out, and into
our grave. If we could stay as vigorous as we are at twelve, it would take about
700 years for half of us to die, and another 700 for the survivors to be reduce
by half again.
\clearpage
\subsection{*Lesson 38 
Water and the traveler}
\par
Contamination of water supplies is usually due to poor sanitation close to water sources, sewage disposal
into the sources themselves, leakage of sewage into distribution systems or contamination with industrial or
farm waste. Even if a piped water supply is safe at its source, it is not always safe by the time it reaches the tap.
Intermittent tap-water supplies should be regarded as particularly suspect.
\par
Travellers on short trips to areas with water supplies of uncertain quality should avoid drinking tap-water,
or untreated water from any other source. It is best to keep to hot drinks, bottled or canned drinks of
well-known brand names-international standards of water treatment are usually followed at bottling plants.
Carbonated drinks are acidic, and slightly safer. Make sure that all bottles are opened in your presence, and that
their rims are clean and dry.
\par
Boiling is always a good way of treating water. Some hotels supply boiled water on request and this can
be used for drinking, or for brushing teeth. Portable boiling elements that can boil small quantities of water are
useful when the right voltage of electricity is available. Refuse politely any cold drink from an unknown
source.
\par
Ice is only as safe as the water from which it is made, and should not be put in drinks unless it is known to
be safe. Drinks can be cooled by placing them on ice rather than adding ice to them.
\par
Alcohol may be a medical disinfectant, but should not be relied upon to sterilize water. Ethanol is more
effective at a concentration of 50-70 percent; below 20 per cent, its bactericidal action is negligible. Spirits
labeled 95 proof contain only about 47 per cent alcohol. Beware of methylated alcohol, which is very
poisonous, and should never be added to drinking water.
\par
If no other safe water supply can be obtained, tap water that is too hot to touch can be left to cool and is
generally safe to drink. Those planning a trip to remote areas, or intending to live in countries where drinking
is not readily available, should know about the various possible methods for making water safe.
\clearpage
\subsection{Lesson 39 
What every writer wants}
\par
I have known very few writers, but those I have known, and whom I respected,
confess at once that they have little idea where they arc going when they first set
pen to paper. They have a character, perhaps two, they are in that condition of
eager discomfort which passes for inspiration, all admit radical changes of
destination once the journey has begun; one, to my certain knowledge, spent nine
months on a novel about Kashmir, then reset the whole thing in the Scottish
Highlands. I never heard of anyone making a 'skeleton', as we were taught at
school. In the breaking and remaking, in the timing, interweaving, beginning
afresh, the writer comes to discern things in his material which were not con-
seriously in his mind when he began. This organic process, often leading to
moments of extraordinary self-discovery, is of an indescribable fascination. A
blurred image appears, he adds a brushstroke and another, and it is gone; but
something was there, and he will not rest till he has captured it. Sometimes the
yeast within a writer outlives a book he has written. I have heard of writers who
read nothing but their own books, like adolescents they stand before the mirror,
and still cannot fathom the exact outline of the vision before them. For the same
reason, writers talk interminably about their own books, winkling out hidden
meanings, super-imposing new ones, begging response from those around them.
\par
Of course a writer doing this is misunderstood: he might as well try to explain a
crime or a love affair. He is also, incidentally, an unforgivable bore.
This temptation to cover the distance between himself and the reader, to
study his image in the sight of those who do not know him, can be his undoing:
he has begun to write to please.
\par
A young English writer made the pertinent observation a year or two back
that the talent goes into the first draft, and the art into the drafts that follow. For
this reason also the writer, like any other artist, has no resting place, no crowd or
movement in which he may take comfort, no judgment from outside which can
replace the judgment from within. A writer makes order out of the anarchy of
his heart; he submits himself to a more ruthless discipline than any critic dreamed
of, and when he flirts with fame, he is taking time off from living with himself,
from the search for what his world contains at its inmost point.
\clearpage
\subsection{*Lesson 40 
Waves}
\par
Waves are the children of the struggle between ocean and atmosphere, the ongoing signatures of infinity. Rays
from the sun excite and energize the atmosphere of the earth, awakening it to flow, to movement, to rhythm, to
life. The wind then speaks the message of the sun to the sea and the sea transmits it on through waves – and
ancient, exquisite, powerful message.
\par
These ocean waves are among the earth's most complicated natural phenomena. The basic features
include a crest ( the highest point of the wave), a trough (the lowest point), a height (the vertical distance from
the trough to the crest), a wave length (the horizontal distance between two wave crests), and a period(which is
the time it takes a wave crest to travel one wave length).
\par
Although an ocean wave gives the impression of a wall of water moving in your direction, in actuality
waves move through the water leaving the water about where it was. If the water was moving with the wave,
the ocean and everything on it would be racing in to the shore with obviously catastrophic results.
\par
An ocean wave passing through deep water causes a particle on the surface to move in a roughly circular
orbit, drawing the particle first towards the advancing wave, then up into the wave, then forward with it and
then – as the wave leaves the particles behind – back to its starting point again.
\par
From both maturity to death, a wave is subject to the same laws as any other ‘living' thing. For a time it
assumes a miraculous individuality that, in the end, is reabsorbed into the great ocean of life.
\par
The undulating waves of the open sea are generated by three natural causes: wind, earth movements of
tremors, and the gravitational pull of the moon and the sun. Once waves have bean generated, gravity is the
force that drives them in a continual attempt to restore the ocean surface to a flat plain.
\clearpage
\subsection{Lesson 41
Training elephants}
\par
Two main techniques have been used for training elephants, which we may call respectively the tough and the
gentle. The former method simply consists of setting an elephant to work and beating him until he does what is
expected of him. Apart from any moral considerations this is a stupid method of training, for it produces a
resentful animal who at a later stage may well turn man-killer. The gentle method requires more patience in the
early stages, but produces a cheerful, good-tempered elephant who will give many years of loyal service.
\par
The first essential in elephant training is to assign to the animal a single mahout who will be entirely
responsible for the job. Elephants like to have one master just as dogs do, and are capable of a considerable
degree of personal affection. There are even stories of half-trained elephant calves who have refused to feed
and pined to death when by some unavoidable circumstance they have been deprived of their own trainer. Such
extreme cases must probably be taken with a grain of salt, but they do underline the general principle that the
relationship between elephant and mahout is the key to successful training.
\par
The most economical age to capture an elephant for training is between fifteen and twenty years, for it is
then almost ready to undertake heavy work and can begin to earn its keep straight away. But animals of this
age do not easily become subservient to man, and a very firm hand must be employed in the early stages. The
captive elephant, still roped to a tree,plunges and screams every time a man approaches, and for several days
will probably refuse all food through anger and fear. Sometimes a tame elephant is tethered nearby to give the
wild one confidence, and in most cases the captive gradually quietens down and begins to accept its food. The
next stage is to get the elephant to the training establishment, a ticklish business which is achieved with the aid
of two tame elephants roped to the captive on either side.
\par
When several elephants are being trained at one time it is customary for the new arrival to be placed between
the stalls of two captives whose training is already well advanced. It is then left completely undisturbed with
plenty of food and water so that it can absorb the atmosphere of its new home and see that nothing particularly
alarming is happening to its companions. When it is eating normally its own training begins. The trainer stands
in front of the elephant holding a long stick with a sharp metal point. Two assistants, mounted or tame
elephants, control the captive from either side, while others rub their hands over his skin to the accompaniment
of a monotonous and soothing chant. This if supposed to induce pleasurable sensations in the elephant, and its
effects are reinforced by the use of endearing epithets, such as 'ho ! my son', or 'ho ! my father', or 'my mother',
according to the age and sex of the captive. The elephant is not immediately susceptible to such blandishments,
however, and usually lashes fiercely with its trunk in all directions. These movements are controlled by the
trainer with the metal-pointed stick, and the trunk eventually becomes so sore that the elephant curls it up and
seldom afterwards uses it for offensive purposes.
\clearpage
\subsection{Lesson 42
Recording an earthquake}
\par
An earthquake comes like a thief in the night, without warning. It was necessary, therefore, to invent
instruments that neither slumbered nor slept. Some devices were quite simple. one, for instance, consisted of
rods of various lengths and thicknesses which would stand up on end like ninepins. when a shock came it
shook the rigid table upon which these stood. If it were gentle, only the more unstable rods fell. If it were
severe, they all fell. Thus the rods by falling, and by the direction in which they fell, recorded for the
slumbering scientist the strength of a shock that was too weak to waken him and the direction from which it
came.
\par
But instruments far more delicate than that were needed if any really serious advance was to be made. The
ideal to be aimed at was to devise an instrument that could record with a pen on paper the movements, of the
ground or of the table, as the quake passed by. While I write my pen moves, but the paper keeps still. With
practice, no doubt, I could in time learn to write by holding the still while the paper moved. That sounds a silly
suggestion, but that was precisely the idea adopted in some of the early instruments (seismometers) for
recording earthquake waves. But when table, penholder and paper are all moving how is it possible to write
legibly? The key to a solution of that problem lay in an everyday observation. Why does a person standing in a
bus or train tend to fall when a sudden start is made? It is because his feet move on, but his head stays still. A
simple experiment will help us a little further. Tie a heavy weight at the endof a long piece of string. With the
hand held high in the air hold the strings so thatthe weight nearly touches the ground. Now move the hand to
and fro and around but not up and down. It will be found that the weight moves but slightly or not at all.
Imagine a pen attached to the weight in such a way that its point rests upon a piece of paper on the floor.
Imagine an earthquake shock shaking the floor, the paper, you and your hand. In the midst of all this movement
the weight and the pen would be still. But as the paper moved from side to side under the pen point its
movement would be recorded in ink upon its surface. It was upon this principle that the first instruments were
made, but the paper was wrapped round a drum which rotated slowly. As long as all was still the pen drew a
straight line, but while the drum was being shaken the line that the pen was drawing wriggled from side to side.
The apparatus thus described, however, records only the horizontal component of the wave movement, which
is, in fact, much more complicated. If we could actually see the path described by a particle, such as a sand
grain in the rock, it would be more like that of a bluebottle buzzing round the room; it would be up and down,
to and fro and from side to side. Instruments have been devised and can he so placed that all three elements can
be recorded in different graphs.
\par
When the instrument is situated at more than 700 miles from the earthquake centre, the graphic record
shows three waves arriving one after the other at short intervals. The first records the arrival of longitudinal
vibrations. The second marks the arrival of transverse vibrations which travel more slowly and arrive several
minutes after the first. These two have travelled through the earth. It was from the study of these that so much
was learnt about the interior of the earth. The third, or main wave, is the slowest and has travelled round the
earth through the surface rocks.
\clearpage
\subsection{Lesson 43
Are there strangers in space?}
\par
We must conclude from the work of those who have studied the origin of life, that given a planet only
approximately like our own, life is almost certain to start. Of all the planets in our own solar system we arc
now pretty certain the Earth is the only one on which life can survive. Mars is too dry and poor in oxygen,
Venus far too hot, and so is Mercury, and the outer planets have temperatures near absolute zero and
hydrogen-dominated atmospheres. But other suns, stars as the astronomers call them, are bound to have planets
like our own, and as the number of stars in the universe is so vast, this possibility becomes virtual certainty.
There are one hundred thousand million stars in our own Milky Way alone, and then there are three thousand
million other Milky Ways, or Galaxies, in the universe. So the number of stars that we know exist is estimated
at about 300 million million million.
\par
Although perhaps only 1 per cent of the life that has started somewhere will develop into highly complex
and intelligent patterns, so vast is the number of planets that intelligent life is bound to be a natural part of the
universe.
\par
If then we are so certain that other intelligent life exists in the universe, why have we had no visitors from
outer space yet ? First of all, they may have come to this planet of ours thousands or millions of years ago, and
found our then prevailing primitive state completely uninteresting to their own advanced knowledge.Professor
Ronald Bracewell, a leading American radio-astronomer, argued in Nature that such a superior civilization, on
a visit to our own solar system, may-have left an automatic messenger behind to await the possible awakening
of an advanced civilization. Such a messenger, receiving our radio and television signals, might well
re-transmit them back to its home-planet, although what impression any other civilization would thus get from
us is best left unsaid.
\par
But here we come up against the most difficult of all obstacles to contact with people on other planets--the
astronomical distances which separate us. As a reasonable guess, they might, on an average, be 100 light years
away. (A light year is the distance which light travels at 186,000 miles per second in one year, namely 6
million million miles.) Radio waves also travel at the speed of light, and assuming such an automatic
messenger picked up our first broadcasts of the 1920's, the message to its home planet is barely halfway there.
Similarly, our own Present primitive chemical rockets, though good enough to orbit men, have no chance of
transporting us to the nearest other star, four light years away, let alone distances of tens or hundreds of light
years.
\par
Fortunately, there is a 'uniquely rational way' for us to communicate with other intelligent beings, as
Walter Sullivan has put it in his excellent recent book, We are not alone. This depends on the precise
radio-frequency of the 21-cm wavelength, or 1420 megacycles per second. It is the natural frequency of
emission of the hydrogen atoms in space and was discovered by us in 1951; it must be known to any kind of
radio-astronomer in the universe.
\par
Once the existence of this wave-length had been discovered, it was not long before its use as the uniquely
recognizable broadcasting frequency for interstellar communication was suggested. Without something of this
kind, searching for intelligences on other planets would be like trying to meet a friend in London without a
Pre-arranged rendezvous and absurdly wandering the streets in the hope of a chance encounter.
\clearpage
\subsection{Lesson 44
Patterns of culture}
\par
Custom has not been commonly regarded as a subject of any great moment. The inner workings of our own
brains we feel to be uniquely worthy of investigation, but custom have a way of thinking, is behaviour at its
most commonplace. As a matter of fact, it is the other way around. Traditional custom, taken the world over, is
a mass of detailed behaviour more astonishing than what any one person can ever evolve in individual actions,
no matter how aberrant. Yet that is a rather trivial aspect of the matter. The fact of first-rate importance is the
predominant role that custom plays in experience and in belief, and the very great varieties it may manifest.
\par
No man ever looks at the world with pristine eyes. He sees it edited by a definite set of customs and
institutions and ways of thinking. Even in his philosophical probings he cannot go behind these stereotypes; his
very concepts of the true and the false will still have reference to his particular traditional customs. John
Dewey has said in all seriousness that the part played by custom in shaping the behaviour of the individual as
over against any way in which he can affect traditional custom, is as the proportion of the total vocabulary of
his mother tongue over against those words of his own baby talk that are taken up into the vernacular of his
family. When one seriously studies the social orders that have had the opportunity to develop autonomously,
the figure becomes no more than an exact and matter-off-fact observation. The life history of the individual is
first and foremost an accommodation to the patterns and standards traditionally handed down in his community.
From the moment of his birth the customs into which he is born shape his experience and behaviour. By the
time he can talk, he is the little creature of his culture, and by the time he is grown and able to take part in its
activities, its habits are his habits, its beliefs his beliefs, its impossibilities his impossibilities. Every child that
is born into his group will share them with him, and no child born into one on the opposite side of the globe
can ever achieve the thousandth part. There is no social problem it is more incumbent upon us to understand
than this of the role of custom. Until we are intelligent as to its laws and varieties, the main complicating facts
of human life must remain unintelligible.
\par
The study of custom can be profitable only after certain preliminary propositions have been accepted, and
some of these propositions have been violently opposed. In the first place any scientific study requires that
there be no preferential weighting of one or another of the items in the series it selects for its consideration. In
all the less controversial fields like the study of cacti or termites or the nature of nebulae, the necessary method
of study is to group the relevant material and to take note of all possible variant forms and conditions. In this
way we have learned all that we know of the laws of astronomy, or of the habits of the social insects, let us say.
It is only in the study of man himself that the major social sciences have substituted the study of one local
variation, that of Western civilization.
\par
Anthropology was by definition impossible as long as these distinctions between ourselves and the
primitive, ourselves and the barbarian, ourselves and the pagan, held sway over people's minds. It was
necessary first to arrive at that degree,of sophistication where we no longer set our own belief over against our
neighbour's superstition. It was necessary to recognize that these institutions which are based on the same
premises, let us say the supernatural, must be considered together, our own among the rest.
\clearpage
\subsection{Lesson 45
Of men and galaxies}
\par
In man's early days, competition with other creatures must have been critical.But this phase of our
development is now finished. Indeed, we lack practice and experience nowadays in dealing with primitive
conditions. I am sure that, without modern weapons, I would make a very poor show of disputing the
ownership of a cave with a bear, and in this I do not think that I stand alone. The last creature to compete with
man was the mosquito. But even the mosquito has been subdued by attention to drainage and by chemical
sprays.
\par
Competition between ourselves, person against person, community against community, still persists,
however; and it is as fierce as it ever was.
\par
But the competition of man against man is not the simple process envisioned in biology. It is not a simple
competition for a fixed amount of food determined by the physical environment, because the environment that
determines our evolution is no longer essentially physical. Our environment is chiefly conditioned by the
things we believe. Morocco and California are bits of the Earth in very similar latitudes, both on the west
coasts of continents with similar climates, and probably with rather similar natural resources. Yet their present
development is wholly different, not so much because of different people even, but because of the different
thoughts that exist in the minds of their inhabitants. This is the point I wish to emphasize. The most important
factor in our environment is the state of our own minds.
\par
It is well known that where the white man has invaded a primitive culture the most destructive effects
have come not from physical weapons but from ideas. Ideas are dangerous. The Holy office knew this full well
when it caused heretics to be burned in days gone by. Indeed, the concept of free speech only exists in
our modem society because when you are inside a community you are conditioned by the conventions of the
community to such a degree that it is very difficult to conceive of anything really destructive. It is only
someone looking on from outside that can inject the dangerous thoughts. I do not doubt that it would be
possible to inject ideas into the modern world that would utterly destroy us. I would like to give you an
example, but fortunately I cannot do so. Perhaps it will suffice to mention the nuclear bomb. Imagine the
effect on a reasonably advanced technological society, one that still does not possess the bomb, of making it
aware of the possibility, of supplying sufficient details to enable the thing to be constructed. Twenty or thirty
pages of information handed to any of the major world powers around the year 1925 would have been
sufficient to change the course of world history. It is a strange thought, but I believe a correct one, that twenty
or thirty pages of ideas and information would be capable of turning the present-day world upside down, or
even destroying it. I have often tried to conceive of what those pages might contain, but of course I cannot do
so because I am a prisoner of the present-day world, just as all of you are. We cannot think outside the
particular patterns that our brains are conditioned to, or, to be more accurate, we can think only a very little
way outside, and then only if we are very original.
\clearpage
\subsection{Lesson 46
Hobbies}
\par
A gifted American psychologist has said, 'Worry is a spasm of the emotion; the mind catches hold of something
and will not let it go.' It is useless to argue with the mind in this condition. The stronger the will, the more
futile the task. One can only gently insinuate something else into its convulsive grasp. And if this something
else is rightly chosen, if it is really attended by the illumination of another field of interest, gradually, and often
quite swiftly, the old undue grip relaxes and the process of recuperation and repair begins.
\par
The cultivation of a hobby and new forms of interest is therefore a policy of first importance to a public man.
But this is not a business that can be undertaken in a day or swiftly improvised by a mere command of the will.
\par
The growth of alternative mental interests is a long process. The seeds must be carefully chosen; they must fall
on good ground; they must be sedulously tended, if the vivifying fruits are to be at hand when needed.
\par
To be really happy and really safe, one ought to have at least two or three hobbies, and they must all be real. It
is no use starting late in life to say: 'I will take an interest in this or that.' Such an attempt only aggravates the
strain of mental effort. A man may acquire great knowledge of topics unconnected with his daily work, and yet
hardly get any benefit or relief. It is no use doing what you like, you have got to like what you do. Broadly
speaking, human beings may be divided into three classes: those who are toiled to death, those who are
worried to death, and those who are bored to death. It is no use offering the manual labourer, tired out with a
hard week's sweat and effort, the chance of playing a game of football or baseball on Saturday afternoon. It is
no use inviting the politician or the professional or business man, who has been working or worrying about
serious things for six days, to work or worry about trifling things at the week-end.
\par
As for the unfortunate people who can command everything they want, who can gratify every caprice and lay
their hands on almost every object of desire—for them a new pleasure, a new excitement is only an additional
satiation. In vain they rush frantically round from place to place, trying to escape from avenging boredom by
mere clatter and motion. For them discipline in one form or another is the most hopeful path.
\par
It may also be said that rational, industrious, useful human beings are divided into two classes: first, those
whose work is work and whose pleasure is pleasure; and secondly, those whose work and pleasure are one. Of
these the former are the majority. They have their compensations. The long hours in the office or the factory
bring with them as their reward, not only the means of sustenance, but a keen appetite for pleasure even in its
simplest and most modest forms. But fortune's favoured children belong to the second class. Their life is a
natural harmony. For them the working hours are never long enough. Each day is a holiday, and ordinary
holidays when they come are grudged as enforced interruptions in an absorbing vocation. Yet to both classes
the need of an alternative outlook, of a change of atmosphere, of a diversion of effort, is essential. Indeed, it
may well be that those whose work is their pleasure are those who most need the means of banishing it at
intervals from their minds.
\clearpage
\subsection{Lesson 47
The great escape}
\par
Economy is one powerful motive for camping, since after the initial outlay upon equipment, or through hiring
it, the total expense can be far less than the cost of hotels. But, contrary to a popular assumption, it is far from
being the only one, or even the greatest. The man who manoeuvres carelessly into his five shillings worth of
space at one of Europe's myriad permanent sites may find himself bumping a Bentley. More likely, Ford
Consul will be hub to hub with Renault or Mercedes, but rarely with bicycles made for two.
\par
That the equipment of modern camping becomes yearly more sophisticated is an entertaining paradox for the
cynic, a brighter promise for the hopeful traveler who has sworn to get away from it all. It also provides--and
some student sociologist might care to base his thesis upon the phenomenon--an escape of another kind. The
modern traveller is often a man who dislikes the Splendide and the Bellavista, not because he cannot afford, or
shuns, their meterial comforts, but because he is afraid of them. Affluent he may be, but he is by no means sure
what, to tip the doorman or the chambermaid. Master in his own house, he has little idea of when to say boo to
a maitre d'hotel.*
\par
From all such fears camping releases him. Granted, a snobbery of camping itself, based upon equipment and
techniques, already exists, but it is of a kind that, if he meets it, he can readily understand and deal with. There
is no superior 'they' in the shape of managements and hotel hierarchies to darken his holiday days.
To such motives, yet another must be added. The contemporary phenomenon of motor-car worship is to be
explained not least by the sense of independence and freedom that ownership entails. To this pleasure camping
gives an exquisite refinement.
\par
From one's own front door to home or foreign hills or sands and back again, everything is to hand. Not only
are the means of arriving at the holiday paradise entirely within one's own command and keeping, but the
means of escape from holiday hell (if the beach proves too crowded, the local weather too inclement) are there,
outside--or, as likely, part of--the tent.
\par
Idealists have objected to the practice of camping, as to the packaged tour, that the traveller abroad thereby
denies himself the opportunity of getting to know the people of the country visited. Insularity and
self-containment, it is argued, go hand in hand. The opinion does not survive experience of a popular
Continental camping place. Holiday hotels tend to cater for one nationality of visitors especially, sometimes
exclusively. Camping sites, by contrast, are highly cosmopolitan. Granted, a preponderance of Germans is a
characteristic that. seems common to most Mediterranean sites; but as yet there is no overwhelmingly
specialized patronage. Notices forbidding the open-air drying of clothes, or the use of water points for car
washing, or those inviting 'our camping friends' to a dance or a boat trip are printed not only in French or
Italian or Spanish, but also in English, German and Dutch. At meal times the odour of sauerkraut vies with that
of garlic. The Frenchman's breakfast coffee competes with the Englishman's bacon and eggs.
\par
Whether the remarkable growth of organized camping means the eventual death of the more independent kind
is hard to say. Municipalities naturally want to secure the campers' site fees and other custom. Police are wary
of itinerants who cannot be traced to a recognized camp boundary or to four walls. But most probably it will all
depend upon campers themselves: how many heath fires they cause, how much litter they leave, in short,
whether or not they wholly alienate landowners and those who live in the countryside. Only good scouting is
likely to preserve the freedoms so dear to the heart of the eternal Boy Scout.
\clearpage
\subsection{*Lesson 48 Planning a share portfolio}
\par
There is no shortage of tipsters around offering ‘get-rich-quick' opportunities. But if you are a serious private
investor, leave the Las Vegas mentality to those with money to fritter. The serious investor needs a proper
‘portfolio' – a well-planned selection of investments, with a definite structure and a clear aim. But exactly how
does a newcomer to the stock market go about achieving that?
\par
Well, if you go to five reputable stock brokers and ask them what you should do with your money, you're
likely to get five different answers, -- even if you give all the relevant information about your age, family,
finances and what you want from your investments. Moral? There is no one ‘right' way to structure a portfolio.
However, there are undoubtedly some wrong ways, and you can be sure that none of our five advisers would
have suggested sinking all (or perhaps any ) of your money into Periwigs.
\par
So what should you do? We'll assume that you have sorted out the basics – like mortgages, pensions,
insurance and access to sufficient cash reserves. You should then establish your own individual aims. These are
partly a matter of personal circumstances, partly a matter of psychology.
\par
For instance, if you are older you have less time to recover from any major losses, and you may well wish
to boost your pension income. So preserving your capital and generating extra income are your main priorities.
In this case, you'd probably construct a portfolio with some shares(but not high risk ones), along with gifts,
cash deposits, and perhaps convertibles or the income shares of split capital investment trusts.
\par
If you are younger, and in a solid financial position, you may decide to take an aggressive approach – but
only if you're blessed with a sanguine disposition and won't suffer sleepless nights over share prices. If you
recognize yourself in this description, you might include a couple of heady growth stocks in your portfolio,
alongside your more pedestrian investments. Once you have decides on your investment aims, you can then
decide where to put your money. The golden rule here is spread your risk – if you put all of your money into
Periwigs International, you're setting yourself up as a hostage to fortune.
\clearpage
\end{document}